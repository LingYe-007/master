%---------------------------------------------------------------------------%
%-                                Packages                                 -%
%---------------------------------------------------------------------------%
% ctex + xeCJK 环境下不再需要 CJK 宏包(避免与 xeCJK 冲突)
% \usepackage{CJK,CJKnumb}
\usepackage{url}
% 正文引用只显示文献编号,右上角上标;参考文献表仍按 GB/T 7714-2015 格式
\usepackage[backend=biber,sorting=none,gbpub=false,bibstyle=gb7714-2015,citestyle=numeric,
gbpunctin=false,gbnamefmt=lowercase,
]{biblatex}
% 引用为正文内 [x] 格式(如 [1] [2]),非上标;仅输出编号,避免 gb7714-2015 输出作者名
\DeclareCiteCommand{\cite}[\mkbibbrackets]
  {\usebibmacro{citeindex}}
  {\usebibmacro{citeindex}%
   \printfield{labelnumber}}
  {\multicitedelim}
  {}
\providecommand*{\mkbibbrackets}[1]{[#1]}

\usepackage{amsmath, amsfonts, amssymb}
\usepackage{booktabs}
\usepackage{fontspec}
\usepackage{threeparttable}
\usepackage{multicol}
\usepackage{multirow}
\usepackage{fancyhdr}
\usepackage{latexsym}
\usepackage{mathrsfs}
% \usepackage{wasysym}  % 字体文件缺失,暂时注释
\usepackage{enumerate}
\usepackage{titletoc}
\usepackage{titlesec}
% 设置章节标题到页眉
\makeatletter
\renewcommand{\sectionmark}[1]{\markboth{#1}{}}
\newcommand{\StartToc}[1]{\@starttoc{#1}}
\let\saved@numberline\numberline
\makeatother

\newlength{\titlelabelsep}
\setlength{\titlelabelsep}{1em}

% 一级标题(章):三号黑体,水平居中,段前段后正常
\titleformat{\section}
  {\normalfont\sanhao\bfseries\heiti\centering}
  {}                                  % 编号由标题文字自带(如“第一章”),不重复输出
  {0pt}
  {}
\titleformat{name=\section,numberless}
  {\normalfont\sanhao\bfseries\heiti\centering}
  {}
  {0pt}
  {}
\titlespacing*{\section}{0pt}{\baselineskip}{0.5\baselineskip}

% 二级标题(节):黑体四号,居左,段前段后正常
\titleformat{\subsection}
  {\normalfont\sihao\bfseries\heiti}
  {\thesubsection}
  {\titlelabelsep}
  {}
\titleformat{name=\subsection,numberless}
  {\normalfont\sihao\bfseries\heiti}
  {}
  {0pt}
  {}
% 标题前后间距(按需更紧凑)
\titlespacing*{\subsection}{0pt}{0.6\baselineskip}{0.25\baselineskip}

% 三级标题(条):小四黑体,居左,无额外段落间距
\titleformat{\subsubsection}
  {\normalfont\xiaosihao\bfseries\heiti}
  {\thesubsubsection}
  {\titlelabelsep}
  {}
\titleformat{name=\subsubsection,numberless}
  {\normalfont\xiaosihao\bfseries\heiti}
  {}
  {0pt}
  {}
\titlespacing*{\subsubsection}{0pt}{0.35\baselineskip}{0.15\baselineskip}
\usepackage{epstopdf}
\usepackage{caption}
\usepackage{graphicx, subfig}
\usepackage{float}
\floatplacement{figure}{H}
\floatplacement{table}{H}
\usepackage{fancyhdr} 
% 页面设置已在 main.tex 中统一设置,此处删除避免冲突
%---------------------------------------------------------------------------%
%-                                  Fonts                                  -%
%---------------------------------------------------------------------------%
\newcommand{\song}{\CJKfamily{song}}    % 宋体   (simsun.ttf)
\makeatletter
\@ifundefined{songti}{\newcommand{\songti}{\CJKfamily{song}}}{}
\@ifundefined{heiti}{\newcommand{\heiti}{\CJKfamily{hei}}}{}
\makeatother
\newcommand{\fs}{\CJKfamily{fs}}        %仿宋体  (simfs.ttf)
\newcommand{\kai}{\CJKfamily{kai}}      % 楷体   (simkai.ttf)
\newcommand{\hei}{\CJKfamily{hei}}      % 黑体   (simhei.ttf)
\newcommand{\li}{\CJKfamily{li}}        % 隶书   (simli.ttf)
%---------------------------------------------------------------------------%
%-                                  Sizes                                  -%
%---------------------------------------------------------------------------%
\newcommand{\chuhao}{\fontsize{42pt}{\baselineskip}\selectfont}
\newcommand{\xiaochuhao}{\fontsize{36pt}{\baselineskip}\selectfont}
\newcommand{\yihao}{\fontsize{28pt}{\baselineskip}\selectfont}
\newcommand{\erhao}{\fontsize{21pt}{\baselineskip}\selectfont}
\newcommand{\xiaoerhao}{\fontsize{18pt}{\baselineskip}\selectfont}
\newcommand{\sanhao}{\fontsize{16pt}{18pt}\selectfont}   % 三号,标题用
\newcommand{\sihao}{\fontsize{14pt}{\baselineskip}\selectfont}
\newcommand{\xiaosihao}{\fontsize{12pt}{14.4pt}\selectfont} % 小四,标题用
\newcommand{\wuhao}{\fontsize{10.5pt}{\baselineskip}\selectfont}
\newcommand{\xiaowuhao}{\fontsize{9pt}{\baselineskip}\selectfont}
\newcommand{\liuhao}{\fontsize{7.875pt}{\baselineskip}\selectfont}
\newcommand{\qihao}{\fontsize{5.25pt}{\baselineskip}\selectfont}
%---------------------------------------------------------------------------%
%-                               Chinesization                             -%
%---------------------------------------------------------------------------%
\newtheorem{theorem}{\hskip 2em定理}[section]
\newtheorem{definition}{\hskip 2em定义}[section]
\newtheorem{exam}{\hskip 2em例}[section]
\newtheorem{proof}{{\it \hskip 2em\textbf{证明}}}
\newtheorem{solution}{{\it \hskip 2em\textbf{解}}}
\renewcommand{\tablename}{\songti 表}
\renewcommand{\figurename}{\song 图}
\renewcommand{\refname}{\centerline {参考文献}}
% 目录、图目录、表目录标题:黑体三号,居中
\renewcommand{\contentsname}{\centerline{\sanhao\heiti 目~~~~录}}
\renewcommand{\listfigurename}{\centerline{\sanhao\heiti 图~~~~目~~~~录}}
\renewcommand{\listtablename}{\centerline{\sanhao\heiti 表~~~~目~~~~录}}
\renewcommand{\thefootnote}{\arabic{footnote}}
\renewcommand{\theequation}{\thesection.\arabic{equation}}
% 图表编号格式:按章编号,使用短横线(如"图3-1"、"表2-1")
\renewcommand{\thefigure}{\thesection-\arabic{figure}}
\renewcommand{\thetable}{\thesection-\arabic{table}}
%---------------------------------------------------------------------------%
%-                                公式随章编号                               -%
%---------------------------------------------------------------------------%
\date{}
\makeatletter
\renewcommand*\l@chapter[2]{%
 \ifnum \c@tocdepth >\m@ne
   \addpenalty{-\@highpenalty}%
   \vskip 0.4em \@plus\p@
   \setlength\@tempdima{0em}%
   \begingroup
     \parindent \z@ \rightskip \@pnumwidth
     \parfillskip -\@pnumwidth
     \leavevmode \bfseries
     \advance\leftskip\@tempdima
     #1\nobreak\leaders\hbox{$\m@th
       \mkern \@dotsep mu\hbox{.}\mkern \@dotsep
       mu$}\hfill \nobreak\hb@xt@\@pnumwidth{\hss #2}\par
     \penalty\@highpenalty
   \endgroup
 \fi}
\date{}
\numberwithin{equation}{section}
\numberwithin{figure}{section}
\numberwithin{table}{section}
%---------------------------------------------------------------------------%
%-                              Tabular Option                             -%
%---------------------------------------------------------------------------%
\newcommand{\tabincell}[2]{\begin{tabular}{@{}#1@{}}#2\end{tabular}}
%---------------------------------------------------------------------------%
%-                              Caption Option                             -%
%---------------------------------------------------------------------------%
% 图表标题格式:五号宋体,紧凑间距
\DeclareCaptionFont{nwucaption}{\wuhao\songti}
\captionsetup[figure]{labelsep=space,font={nwucaption},skip=3pt,belowskip=3pt}
\captionsetup[table]{labelsep=space,font={nwucaption},skip=3pt,aboveskip=3pt}
% 减少图表与正文的间距
\setlength{\floatsep}{6pt plus 2pt minus 2pt}      % 两个浮动体之间的距离
\setlength{\textfloatsep}{8pt plus 2pt minus 2pt}  % 浮动体与正文之间的距离
\setlength{\intextsep}{6pt plus 2pt minus 2pt}     % 浮动体在文本中的间距
%---------------------------------------------------------------------------%
%-                             Reference Resource                          -%
%---------------------------------------------------------------------------%
% \addbibresource{Tex/reference.bib}  % 已在 main.tex 中配置
%---------------------------------------------------------------------------%
%-                                Page Control                             -%
%---------------------------------------------------------------------------%
% 版心由 main.tex 中 \geometry 统一设置,此处不再重复以免冲突
% 目录格式:符合中文学术论文标准
% 目录标题:居中,三号黑体(已在上面设置 \contentsname)

% 第一级(章):黑体四号字,顶格;不显示章前数字(仅显示“第一章”“第二章”等标题文字)
\titlecontents{section}[0em]{\fontsize{14pt}{16.8pt}\selectfont\heiti\bfseries\renewcommand{\numberline}[1]{}}{}{}%
{\titlerule*[0.7pc]{$\cdot$}\contentspage}[\vspace{0.5pt}\let\numberline\saved@numberline]

% 第二级(节):宋体小四号字,缩进 2.5em(1.1 等靠后)
\titlecontents{subsection}[2.5em]{\fontsize{12pt}{14.4pt}\selectfont\songti}{\contentslabel{2.5em}\ }%
{}{\titlerule*[0.7pc]{$\cdot$}\contentspage}[\vspace{0.5pt}]

% 第三级(条):宋体小四号字,缩进 3.8em(相对节再靠后)
\titlecontents{subsubsection}[3.8em]{\fontsize{12pt}{14.4pt}\selectfont\songti}{\contentslabel{3em}\ }%
{}{\titlerule*[0.7pc]{$\cdot$}\contentspage}[\vspace{0.5pt}]

% 第四级(段):宋体小四号字,缩进 4em
\titlecontents{paragraph}[4em]{\fontsize{12pt}{14.4pt}\selectfont\songti}{\contentslabel{3.5em}\ }%
{}{\titlerule*[0.7pc]{$\cdot$}\contentspage}[\vspace{0.5pt}]
% 图目录和表目录格式:五号宋体,紧凑间距,与目录格式一致
\makeatletter
\renewcommand*\l@figure{\@dottedtocline{1}{0em}{2.3em}}
\renewcommand*\l@table{\@dottedtocline{1}{0em}{2.3em}}
\makeatother
\renewcommand{\baselinestretch}{1.5}  % 正文行间距1.5倍,符合格式要求
% 设置段落首行缩进
\usepackage{indentfirst}  % 确保每个段落首行都缩进
\setlength{\parindent}{2em}  % 每个段落首行缩进2个字符