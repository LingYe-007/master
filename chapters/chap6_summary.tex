\newpage
\section*{第六章\quad 总结与展望}
\addcontentsline{toc}{section}{第六章\quad 总结与展望}
\setcounter{section}{6} \setcounter{subsection}{0}

\label{chap:conclusion}

\subsection{全文总结}
\label{sec:summary}
随着数据隐私法规的完善,传统集中式推荐面临严峻合规挑战。联邦学习为打破数据孤岛提供了可行路径,但将异构图神经网络应用于联邦推荐,仍需克服\textbf{数据分布异构(Non-IID)}、\textbf{冷启动交互稀疏}及\textbf{边缘通信受限}三大挑战。本文围绕联邦异构图推荐系统的\textbf{“效果增益”}与\textbf{“效率优化”}两个核心维度展开研究,主要成果总结如下:

\begin{enumerate}[wide, labelwidth=!, labelindent=\parindent]
    \item \textbf{提出了基于属性-结构双视图对比学习的联邦推荐框架 (FedASCL),缓解了冷启动与 Non-IID 问题。}
    为解决数据稀疏与分布异构问题,本文构建了融合显式属性与隐式结构的双视图学习范式。该方法通过最大化互信息,将丰富的属性语义迁移至稀疏的交互空间,以弥补新用户特征缺失;同时引入全局类别原型作为语义锚点,校正本地模型因数据偏斜导致的语义漂移,提升了模型在异构环境下的泛化能力。

    \item \textbf{提出了基于语义感知的模型参数压缩策略,明显降低了通信开销。}
    为缓解异构图模型的通信压力,本文设计了"结构剪枝与数值量化协同"的压缩方案。该方案首先利用注意力机制动态剔除冗余元路径,在降低维度的同时实现"语义去噪"以增强鲁棒性;其次结合残差补偿量化机制,利用本地误差累积策略处理梯度更新。实验表明,该策略在将通信开销降低约一个数量级($20\times$)的同时,依然保持了与原始模型相当的推荐精度。

    \item \textbf{设计并实现了面向学术场景的联邦推荐系统,验证了算法的工程可行性。}
    为解决联邦推荐系统落地困难的问题,本文构建了支持异构设备协同的联邦论文推荐系统。系统采用端云协同的分层架构设计,完成了数据存储层、算法引擎层、业务服务层、业务展示层的解耦,保障了系统的可扩展性与可维护性。系统集成了 FedASCL 框架的核心算法模块,包括本地训练器、联邦协调器、安全聚合器等关键组件,完成了完整的联邦学习训练闭环。在推荐服务方面,系统完成了分级推荐策略,能够根据用户交互历史自动选择结构视图或属性视图进行推荐,解决了冷启动问题。通过系统功能测试验证,系统在功能完备性、推荐准确性以及隐私保护方面均达到了预期目标,为异构图联邦学习在实际学术推荐场景中的应用提供了工程范例。
\end{enumerate}

本文在保证隐私的同时,平衡了联邦异构图推荐系统的精度与效率,并通过系统实现验证了算法的工程可行性,为边缘设备上的实际部署提供了参考方案。

\subsection{研究不足与未来展望}
\label{sec:outlook}
尽管本文取得了一定进展,但受限于理论深度与场景复杂性,仍存在改进空间。未来工作将从以下三个维度展开:

\begin{enumerate}[wide, labelwidth=!, labelindent=\parindent]
    \item \textbf{时序动态性与持续学习 (Temporal Dynamics \&
    Continual Learning)}:
    本文目前基于静态图快照进行研究,尚未充分考虑用户兴趣随时间演变的特性。未来需探索结合动态图神经网络与联邦持续学习,设计适应用户兴趣漂移的增量更新机制,在捕捉即时兴趣的同时避免模型更新过程中的灾难性遗忘问题。

    \item \textbf{图结构隐私防御 (Defense against Graph Reconstruction)}:
    尽管本文使用了梯度聚合保护隐私,但在极端攻击下,图结构信息仍面临重构风险。未来需研究更强有力的防御手段,例如在元路径聚合过程中引入本地差分隐私 (LDP) 机制,或结合基于可信执行环境 (TEE) 的安全协议,以构建更严密的隐私防护墙。

    \item \textbf{跨域联邦与知识迁移 (Cross-Domain Federated Recommendation)}:
    当前工作主要集中在同域场景下的推荐优化。未来将探索在特征空间不重叠、数据分布差异巨大的跨域场景下,如何利用重叠用户或共享特征完成知识迁移,解决目标域数据极度匮乏的问题,拓展联邦推荐的应用边界。
\end{enumerate}