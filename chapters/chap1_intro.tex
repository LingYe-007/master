\newpage
\pagenumbering{arabic}\setcounter{page}{1}
% 页眉页脚已在 main.tex 中统一设置,此处不需要重复设置
\pagestyle{fancy}
% \usepackage{indentfirst}  % 已在 option.tex 中加载,此处不需要
\section{第一章\quad 绪论}
\label{chap:intro}
\setcounter{section}{1} \setcounter{subsection}{0}

% ----------------------------------------------------------------------
% 第一部分:引言
% ----------------------------------------------------------------------
% 本章节将对本文的研究背景进行介绍,引出联邦推荐目前面临的问题与挑战,阐述本文研究工作的意义。本章还会对近年来国内外研究现状进行介绍,并简要阐述本文所做研究工作。最后,给出本文的总体论文框架。

% ----------------------------------------------------------------------
% 第二部分:研究背景
% ----------------------------------------------------------------------
\subsection{研究背景与意义}

移动互联网技术的快速发展和智能终端设备的广泛普及,使得数字社会的信息流呈现爆发式增长。推荐系统(Recommender System)作为解决"信息过载"与实现"用户-内容"精准匹配的核心技术,已广泛应用于电子商务、社交网络及内容聚合平台,影响用户体验与商业效果。

然而,在追求推荐算法精准度的同时,数据隐私安全问题日益凸显。近年来,随着欧盟《通用数据保护条例》(GDPR)\cite{gdpr2016}、中国《中华人民共和国个人信息保护法》(PIPL)\cite{pipl2021}及《数据安全法》等法律法规的相继颁布,数据主权与隐私合规成为必须满足的约束。在此背景下,联邦学习(Federated Learning, FL)\cite{wu2021fedgnn}作为一种“数据不动模型动”的分布式隐私计算方式,被应用在推荐领域。它允许参与方在本地保留原始数据的同时,仅通过加密交互模型参数或梯度协同构建全局模型,为完成隐私保护下的智能推荐提供了一种可行思路。

然而,由于联邦学习的特殊性,联邦推荐系统的发展仍然面临冷启动、推荐效果变差、通信成本高、落地难等问题\cite{ye2023heterogeneous, zhang2024ifedrec}。具体如下:

\textbf{(1)严重的冷启动问题(Cold Start)与数据异质性导致的模型性能瓶颈(Non-IID,即数据非独立同分布)}:在传统集中式系统中,新用户即便无交互记录,系统仍可利用其属性特征匹配相似用户群\cite{qiao2018coldstart}。然而在联邦架构下,新用户数据驻留本地,服务器无法利用全局属性信息进行相似度匹配\cite{zhang2024ifedrec},由于缺乏历史交互数据驱动梯度更新,本地模型因缺乏历史交互数据而难以学习,难以通过属性语义弥补拓扑信息的缺失。此外,联邦环境中普遍存在的数据非独立同分布(Non-IID)特性\cite{ye2023heterogeneous}进一步加剧了推荐性能的下降。在传统联邦推荐架构下,各客户端的本地数据分布(如用户交互偏好、物品类别分布)存在显著的统计异质性(Statistical Heterogeneity),其模型参数(梯度)朝着不同的局部最优方向更新。当服务器对这些"方向冲突"的参数进行聚合时,会引发严重的"客户端漂移"(Client Drift)现象\cite{tan2023fedstar},使得最终生成的全局模型难以收敛到全局最优解,从而导致模型无法精准适配特定用户的个性化需求,整体推荐准确率明显低于集中式训练。

\textbf{(2)高昂的通信与计算开销(Communication \& Computation Cost)}:异构图推荐模型的核心参数是节点嵌入矩阵,大小随着用户和物品数量线性增长,推荐模型的体积往往高达数百MB\cite{jiang2022fedmp, khan2025hufe},联邦学习通常需要数百轮的迭代才能收敛,会导致网络拥塞,训练速度慢\cite{reisizadeh2020fedpaq}。

\textbf{(3)系统工程落地困难(System Deployment)}:现阶段,多数工作仍在算法层面,系统落地案例较少。尽管学术界已产出大量成果,但在实际工业生产环境中的落地研究仍有较大空白。头部企业(如阿里巴巴、苹果、Google)已率先在特定模块实现了联邦推荐的落地,但大规模普及仍面临数据异质性、通信算力开销、隐私安全防御、评估监控等主要问题。目前学术界大多聚焦于算法层面的理论创新,而针对系统架构设计与工程落地的研究却相对匮乏\cite{guo2024fedgnn},缺乏完整的端到端系统实现和工程验证。

尽管联邦学习为推荐系统中的隐私保护提供了可行思路,但上述冷启动与数据异质性导致的性能瓶颈、高昂的通信与计算开销以及系统落地困难,仍制约着联邦推荐在真实场景中的应用。因此,解决这几类问题、推动联邦推荐从理论研究走向实际应用,具有重要的理论价值和实践意义。针对上述问题,本文提出以下解决思路:

\textbf{(1)针对冷启动与数据异质性问题}:本文提出的 FedASCL 框架通过构建属性语义视图与交互结构视图,利用跨视图对比学习机制将丰富的属性语义信息迁移至稀疏的交互空间,从而弥补新用户特征缺失;同时引入全局语义原型对齐机制,通过约束本地模型表征向全局一致的语义原型靠拢,纠正因数据异质性导致的客户端模型漂移现象,提升模型在Non-IID环境下的泛化能力(详见第三章)。

\textbf{(2)针对通信开销问题}:本文提出基于语义感知的模型参数压缩策略,采用动态元路径选择器与残差梯度量化(详见第四章),在保证模型语义完整性的同时,将通信开销降低约一个数量级($20\times$)。

\textbf{(3)针对系统落地困难问题}:本文设计并实现了支持异构设备协同的联邦论文推荐系统,采用端云协同的分层架构并集成 FedASCL 与压缩策略,完成了从算法到系统的工程验证(详见第五章)。

% ----------------------------------------------------------------------
% 第三部分:现状
% ----------------------------------------------------------------------
% --- 1.2 相关研究现状 ---
\subsection{相关研究现状}

本章按冷启动、数据异质性、通信性能三方面组织相关研究现状:先介绍基于图神经网络的推荐算法研究,再分别综述上述三方面的进展,并分析现有研究的不足,为本文工作提供理论基础。

\subsubsection{基于图神经网络的推荐算法研究}
推荐系统作为解决信息过载的核心技术,其发展历程经历了从基于统计学的传统协同过滤(Collaborative Filtering, CF),到基于深度学习的神经协同过滤(Neural CF),再到如今基于图学习的结构化推荐三个阶段。早期的矩阵分解(MF)技术将用户与物品映射为潜在向量,通过内积运算拟合交互得分,但受限于线性组合的表达能力,难以捕捉复杂的非线性特征\cite{wang2019ngcf}。图神经网络(GNN)\cite{kipf2016semi}的提出使得研究者发现推荐系统中的"用户-物品"交互数据天然构成了图结构,利用 GNN 在非欧几里得空间中的"嵌入传播(Embedding Propagation)"能力,可以有效捕捉高阶协同信号(High-order Connectivity)\cite{he2020lightgcn}。根据对图结构建模复杂度的不同,该领域的发展主要经历了基于图的高阶协同传播与基于元路径的语义建模两个阶段。

(1)\textbf{基于图的高阶协同传播与结构简化:} 这一阶段的研究核心在于利用图卷积网络(Graph Convolutional Network, GCN)捕捉用户与物品间潜在的高阶协同信号(High-order Connectivity)。NGCF (Neural Graph Collaborative Filtering)\cite{wang2019ngcf} 率先奠定了这一领域的基石,它通过"嵌入传播"层显式地模拟了用户偏好的高阶传递过程,克服了传统协同过滤(CF)无法利用多跳邻居信息的缺陷。

鉴于仅依赖交互数据的稀疏性局限,KGAT (Knowledge Graph Attention Network)\cite{wang2019kgat} 进一步将协同传播范式拓展至外部知识图谱(Knowledge Graph, KG)。该模型利用递归注意力机制聚合实体间的语义关联,不仅增强了物品表征的丰富度,也为推荐结果引入了可解释性。

然而,后续研究表明,NGCF 与 KGAT 中沿用自计算机视觉的非线性激活与特征变换带来了不必要的计算负担。LightGCN\cite{he2020lightgcn} 对此进行了理论解构,论证了图推荐中线性聚合的充分性,通过剔除所有非线性操作确立了轻量级图协同过滤的基准。此外,为追求极致的效率与鲁棒性,UltraGCN\cite{mao2021ultragcn} 创新性地跳过了显式的消息传递过程,转而通过约束损失函数逼近无限层图卷积,结合谱图理论(Spectral Graph Theory)抑制了数据噪声,完成了精度与效率的双重突破。

(2)\textbf{基于元路径的语义建模:} 现实世界的交互数据天然构成了包含多种节点类型与边关系的异构信息网络(HIN)\cite{wang2019han}。为了从复杂的网络拓扑中精准捕获用户偏好,HAN\cite{wang2019han} 等经典算法通过元路径引导的注意力机制,完成了异构语义的对齐与聚合,证明了利用异构信息辅助推荐精度的有效性。此外,MAGNN (Masked Graph Attention Network)\cite{fu2020magnn} 指出 HAN 忽略了元路径内部中间节点的特征信息,通过引入路径内的编码器与变换器,完成了对长距离语义依赖的更精细化捕捉,在序列推荐等任务中提升了预测准确率。

尽管基于元路径的方法在一定程度上提升了性能,但其严重依赖人工先验知识\cite{wang2019han}。为解决这一局限,HGT (Heterogeneous Graph Transformer)\cite{hu2020hgt} 提出了元关系(Meta-relation)感知的参数化机制,无需人工定义路径即可自动学习异构子图的动态注意力。然而,上述方法仍难以剔除原始交互数据中广泛存在的噪声(如误点击)。为进一步突破推荐精度的瓶颈,HGSL (Heterogeneous Graph Structure Learning)\cite{zhang2021hgsl} 等算法提出了图结构联合学习的思路。这类方法不再被动地在固定图上进行卷积,而是动态地生成最优的异构图结构,自动识别并强化对推荐目标有益的语义连接,同时抑制噪声干扰。这种端到端的结构优化策略,使得模型能够从纷繁复杂的异构数据中提炼出更纯净的用户意图表征,从而在 Top-K 推荐任务中取得了更优异的预测表现。

基于图神经网络的推荐算法经历了从同构图的高阶传播到异构图的语义对齐,再到多行为复杂交互建模的演进过程。算法架构也从盲目堆砌深层网络,逐渐回归到针对推荐任务特性的线性化、语义化和轻量化设计。这为本文在联邦环境下探索异构图的属性-结构双视图对齐奠定了理论基础。

\subsubsection{联邦推荐系统冷启动研究现状}
如 1.1 节所述,冷启动在联邦推荐场景下更为严峻。针对联邦推荐场景下的冷启动难题,学界开展了很多研究,相关方法大概可归纳为三大类:

(1)\textbf{早期联邦推荐冷启动方法:} 此类方法的核心是将传统集中式推荐算法扩展至联邦场景,初步实现隐私保护与冷启动性能的平衡,为后续研究奠定基础。FedNCF\cite{luo2024perfedrec} 将神经协同过滤(Neural Collaborative Filtering, NCF)扩展至联邦环境,通过多层感知机(MLP)建模用户-物品交互的非线性关系,是联邦推荐领域的经典基线方法。然而,FedNCF仅基于用户-物品交互矩阵进行建模,无法利用图结构中的高阶协同信号,在数据稀疏场景下性能受限。FedGNN\cite{wu2021fedgnn} 突破数据本地化对图结构构建的限制,采用隐私保护的用户-物品图扩展策略,捕捉分布式场景下的高阶协同信号,有效缓解数据稀疏性问题,间接改善冷启动推荐表现。但该方法仅适用于同构图结构,无法处理异构信息网络中的复杂语义关系,且在完全零交互的冷启动场景下,由于缺乏边连接,图神经网络难以学习有效的节点表示。有工作将矩阵分解技术延伸至联邦环境\cite{ye2023heterogeneous},通过随机梯度下降算法实现隐私保护下的协同过滤,但该类方法仅能建模单一交互关系,对冷启动用户的适配能力有限。FedDCSR\cite{zhang2024feddcsr} 进一步提出解耦表示学习方案,实现跨域联邦推荐,借助跨域知识迁移机制,帮助新用户快速建立个性化推荐模型,进一步优化了冷启动场景的适配性能。

(2)\textbf{基于特定技术的联邦推荐冷启动方法:} 随着人工智能技术的发展,学界逐渐将文本特征、注意力机制、大语言模型、元学习等前沿技术与联邦推荐结合,针对性解决冷启动问题,形成了一系列优化方法。
TransFR\cite{transfr2026} 提出可迁移联邦推荐框架,摒弃传统离散项目ID,采用细粒度文本特征作为核心输入,通过预训练语言模型(DistBERT)与联邦适配器微调(FAT)技术,在无交互数据场景下可基于项目描述生成通用语义嵌入,彻底解决ID跨域迁移失效导致的冷启动问题,但该方法依赖预训练模型与完整文本信息,计算开销大且适用场景受限。FedDMR\cite{feddmr2025} 设计包含双层多头注意力与正则化策略的联邦推荐系统,通过交互层与聚合层提取高阶交互表征,利用差异正则化约束本地模型更新不偏离全局模型,缓解数据稀疏导致的过拟合与冷启动性能低下问题,但该方法对属性信息利用不足,对冷启动的针对性有待提升。FELLRec\cite{fellrec2025} 提出相应联邦推荐框架,通过高效计算与存储方案,在保护隐私的前提下利用LLM增强推荐效果,在客户端数据不平衡的冷启动场景中表现出更优鲁棒性,但该方法计算、存储开销巨大,隐私保护机制复杂,难以在边缘设备部署。基于元学习的方法利用其"快速适配"特性,解决模型对新用户、新项目的快速适配问题,MetaFRS\cite{metafrs2025} 将元学习与联邦推荐结合,通过全局共享元初始化参数,利用极少量本地交互数据与邻居社交关系快速适配推荐策略,但该方法依赖少量交互数据与社交关系,适用场景受限。

(3)\textbf{集中式场景通用冷启动方法(对比参考):} 在集中式推荐场景中,基于大语言模型与多模态技术的冷启动方法已取得显著进展,但由于架构差异,难以直接迁移至联邦环境,可作为联邦冷启动方法的对比参考。SaviorRec\cite{saviorrec2025} 聚焦语义-行为对齐,利用多模态编码器生成行为感知语义表示,通过模态-行为对齐块确保语义空间与用户行为空间的一致性;RDNMF\cite{rdnmf2026} 结合BERT与LSTM,处理项目文本信息并捕捉用户偏好动态变化,缓解冷启动偏好缺失问题;MoLoRec\cite{molorec2025} 采用LoRA混合专家策略,通过领域通用与特定模块的动态合并,提升跨域冷启动精度。此类方法的共性局限是:针对集中式场景设计,难以适配联邦隐私约束,且依赖丰富文本、多模态数据,在数据稀疏场景效果受限。

综合上述各类方法可见,当前联邦推荐冷启动研究虽取得一定进展,但仍存在主要不足:部分方法依赖大模型或文本数据,计算开销较大,无法适配边缘设备部署需求;对属性信息的挖掘利用不充分,难以实现完全零交互场景下的有效推荐;多数方法未进行联邦原生设计,难以充分适配联邦架构的隐私约束与数据分布特性;缺乏对异构信息网络语义结构的深度利用。针对上述不足,本文提出的FedASCL框架通过轻量级设计、属性-结构双视图融合、联邦原生设计、异构图语义感知和全局语义原型对齐等机制,有效解决了联邦推荐冷启动问题。

\subsubsection{联邦推荐系统数据异质性研究现状}
如 1.1 节所述,Non-IID 会导致客户端漂移与全局收敛困难\cite{ye2023heterogeneous, tan2023fedstar}。为解决数据异质性导致的模型偏差问题,学术界将部分研究重心转向个性化联邦推荐(Personalized FL)。FedPerGNN\cite{wu2022fedpergnn} 利用元学习机制实现全局模型对本地异质数据的快速适应。该方法通过全局共享的元初始化参数,使模型能够利用极少量本地数据快速适配到本地分布,有效缓解了Non-IID数据导致的性能下降。但该方法主要关注模型参数的个性化,对异构语义结构的利用不足,且在冷启动场景下,由于缺乏足够的本地数据,元学习机制难以有效工作。FedHGNN\cite{yan2024fedhgnn} 通过解耦异构图中的私有与共享元路径,实现了异构语义在隐私约束下的对齐。该方法能够处理异构信息网络中的多类型节点和关系,通过区分私有元路径和共享元路径,在保护隐私的同时实现跨客户端的语义对齐。但在冷启动场景下,由于缺乏交互数据,元路径的语义信息难以有效利用,且该方法未充分利用属性信息来弥补交互数据的缺失。FedProto\cite{tan2022fedproto} 提出联邦原型学习框架,通过在服务器端聚合各客户端的本地原型,生成全局语义原型,客户端利用这些原型对本地表示进行正则化约束。该方法通过原型对齐机制有效缓解了Non-IID数据分布导致的客户端漂移问题,在异构环境下表现优异。但FedProto主要关注原型对齐,对属性信息的利用不足,在完全零交互的冷启动场景下,由于缺乏足够的本地数据来生成有意义的原型,性能受限。

对比学习作为解决数据稀疏与异质分布的方案被引入联邦推荐\cite{wang2025fedpcl},为缓解 Non-IID 数据造成的全局与局部语义割裂提供了新方式。FCCF \cite{wu2023fccf} 率先提出基于双重视角的联邦对比框架,通过对齐全局公共表征与本地个性化表征,有效抑制了噪声梯度的负面影响。PerFedRec++ \cite{luo2024perfedrec} 通过构建多视图协同对比机制,利用结构扰动生成的辅助视图来校正本地数据的分布偏差,验证了跨视图增强在联邦环境下的可行性。然而,现有方法(如FedPerGNN、FedHGNN、FedProto、FCCF)在处理异构信息网络场景时,如何在保护隐私的同时充分利用异构语义结构实现高效的对比学习,目前的方案仍显不足。针对这一不足,本文提出的FedASCL框架通过全局语义原型对齐机制和属性-结构双视图对比学习,有效缓解了Non-IID数据导致的客户端漂移问题,为联邦推荐系统在异构环境下的性能提升提供了新的解决思路。

\subsubsection{联邦推荐系统通信性能研究现状}
如 1.1 节所述,异构图模型参数量大、训练轮次多,通信成为瓶颈。随着联邦推荐向移动端、物联网设备等端侧场景延伸,模型轻量化成为解决端侧算力有限、通信带宽受限的关键方向\cite{jiang2022fedmp, khan2025hufe, reisizadeh2020fedpaq}。

(1)\textbf{模型压缩技术在联邦推荐中的应用:} 模型剪枝和量化是降低通信开销的主要技术手段。模型剪枝通过去除深度神经网络中的冗余参数与结构,以降低计算开销并加速推理过程\cite{jiang2022fedmp}。根据剪枝粒度的不同,主要分为非结构化剪枝和结构化剪枝两类。模型量化是指将神经网络的权重参数和激活值从高精度表示(如 32 位浮点数 FP32)转换为低精度表示(如 8 位整数 INT8 或更低比特)的过程,能够显著降低模型的存储占用和内存带宽需求\cite{khan2025hufe}。

(2)\textbf{联邦推荐中的通信优化方法:} 早期 FedFast \cite{reisizadeh2020fedpaq} 通过采样聚合与激活函数优化确立了高效通信的联邦基准。相关研究逐步兴起:基于稀疏化的通用方法中,Random-k 是一种语义无关的稀疏化方法,每轮随机选择 $k\%$ 的梯度参数进行上传,代表无差别的压缩策略。该方法实现简单,但忽略了参数的重要性,压缩后性能下降明显。Top-k Sparsification\cite{jiang2022fedmp} 基于梯度幅值的稀疏化方法,仅保留绝对值最大的 $k\%$ 参数。相比Random-k,Top-k能够保留更重要的梯度信息,但在异构图推荐中,该方法忽略了元路径的语义结构,难以实现语义感知的压缩。基于量化的通用方法中,QSGD\cite{alistarh2017qsgd} 是标准的随机梯度量化方法,将梯度随机量化为低比特表示,但不包含结构剪枝和残差补偿机制。该方法能够显著降低通信开销,但量化误差会随训练轮次累积,影响模型收敛精度。联邦推荐专用压缩方法中,FedLiteRec \cite{zhu2023fedlp} 采用结构化剪枝与低秩分解结合的策略,在保留核心交互语义的同时,将模型参数量压缩 60\%,同时通过梯度量化降低传输开销。但该方法采用粗粒度的剪枝策略,未考虑异构图中的元路径语义结构,在异构语义场景下难以实现精准压缩。FedQRec \cite{liu2024adaptive} 提出自适应比特量化机制,根据参数重要性动态分配量化比特,核心语义参数采用 8-bit 高精度量化,冗余参数采用 4-bit 低精度量化,在压缩率达 75\% 的情况下保持推荐精度损失低于 3\%。但该方法主要关注参数级别的量化,未考虑语义通道级别的压缩策略。FedDistillRec \cite{yi2023fedlpq} 引入联邦知识蒸馏框架,服务器端训练高精度教师模型,客户端仅部署轻量化学生模型,通过蒸馏损失对齐师生模型表征,既降低端侧计算负担,又缓解 Non-IID 导致的性能下降。但知识蒸馏需要额外的训练开销,且学生模型的压缩比受限于模型架构设计。

然而,在模型轻量化方面,还未发现基于元路径的剪枝优化方面的研究,现有方法(如FedLiteRec、FedQRec、FedDistillRec、Random-k、Top-k、QSGD)往往采用粗粒度的压缩策略,难以在异构语义场景下实现精准的语义感知压缩。本文提出的语义感知压缩策略采用动态元路径选择器与残差梯度量化(见第四章),在约20倍压缩倍数下保持推荐精度并将通信开销降低约一个数量级(见第四章表\ref{tab:performance_comparison}、表\ref{tab:communication_cost})。

\subsubsection{现有研究总结与不足}
综合上述研究可见,针对联邦推荐系统面临的冷启动、数据异质性、通信开销等问题,学术界已开展了大量研究并取得了一定进展。在冷启动问题上,从早期的 FedNCF、FedGNN 等基础方法,到基于文本特征、注意力机制、大语言模型、元学习等前沿技术的优化方法,逐步提升了冷启动场景下的推荐性能;在数据异质性问题上,通过个性化联邦推荐、对比学习等机制,有效缓解了Non-IID数据分布导致的客户端漂移问题;在通信开销问题上,通过模型剪枝、量化、知识蒸馏等技术,显著降低了联邦推荐的通信成本。

然而,现有研究在解决上述几类问题方面仍存在诸多不足。在冷启动问题上,部分方法(如TransFR、FELLRec)依赖大模型或文本数据,计算开销较大,无法适配边缘设备部署需求;对属性信息的挖掘利用不充分,难以实现完全零交互场景下的有效推荐;缺乏对异构信息网络语义结构的深度利用,未能充分挖掘多类型节点与关系中的有效信息。在数据异质性问题上,现有方法(如FedPerGNN、FedHGNN、FCCF)在处理异构场景下的多行为数据时,如何在保护复杂语义隐私的同时实现高效的自适应对比增强,目前的方案仍显匮乏;由于联邦学习的物理隔离性,单个客户端仅持有局部交互子图,导致现有的对比机制难以捕获跨域的长程拓扑关联,存在全局结构感知受限的缺陷。在通信开销问题上,现有压缩方法(如FedLiteRec、FedQRec、FedDistillRec)采用粗粒度的压缩策略,难以在异构语义场景下实现精准的语义感知压缩;在隐私噪声与通信压缩的双重约束下,如何保持对比学习生成的表征具有足够的辨别力,依然是制约联邦图对比推荐系统落地的主要矛盾。

针对上述不足,本文提出的 FedASCL 框架在隐私合规的同时完成高性能的个性化推荐。具体而言:

\textbf{(1)冷启动问题}:相比FedGNN、FedProto等现有方法,本文通过属性-结构双视图对比学习,在冷启动场景下性能提升\textbf{12-15\%}(见第三章表\ref{tab:cold_start_res})。

\textbf{(2)数据异质性问题}:相比FedPerGNN、FedHGNN等现有方法,本文通过全局语义原型对齐机制,在高度异构环境($\alpha=0.1$)下性能提升\textbf{5.2\%}(见第三章表\ref{tab:non_iid_data})。

\textbf{(3)通信开销问题}:相比FedLiteRec、FedQRec等现有方法,本文通过语义感知压缩策略,在约20倍压缩倍数下保持了推荐精度,将通信开销降低了约一个数量级(见第四章表\ref{tab:communication_cost})。

\textbf{(4)系统落地问题}:相比现有研究多聚焦于算法层面,本文设计并实现了完整的联邦推荐系统,完成了从算法到系统的工程验证(详见第五章)。
% ----------------------------------------------------------------------
% 第四部分:贡献
% ----------------------------------------------------------------------
\subsection{本文研究工作和主要贡献}

本文研究旨在探讨联邦推荐系统面临的若干关键问题,包括冷启动问题、推荐效果差、通信成本高以及系统实现困难。为了应对这些问题与挑战,本文主要研究包括如下内容:第三章提出 FedASCL 框架解决冷启动和数据异质性问题;第四章提出语义感知压缩策略解决通信开销问题;第五章设计并实现联邦推荐系统解决系统落地问题。

\begin{enumerate}[wide, labelwidth=!, labelindent=\parindent]
    \item 为解决冷启动和非独立同分布导致的推荐效果差问题,本文提出 \textbf{FedASCL 框架},通过跨视图对比学习与全局语义原型对齐解决冷启动与 Non-IID 问题(详见第三章)。
    
    \item 为降低联邦异构图模型的通信开销,本文提出语义感知的模型参数压缩策略,采用动态元路径选择器与残差梯度量化(见第四章),将通信开销降低约一个数量级。
    
    \item 为解决联邦推荐系统落地困难的问题,本文构建了支持异构设备协同的联邦论文推荐系统,采用端云协同架构并集成本地差分隐私模块(详见第五章)。
\end{enumerate}

% ----------------------------------------------------------------------
% 第五部分:结构
% ----------------------------------------------------------------------
\subsection{论文结构}

文章共六章,分别涉及本文研究的背景、意义、相关研究基础、算法模型、系统实现以及总结和展望。文章研究框架如图~\ref{fig:framework} 所示。

\begin{figure}[H]
    \centering
    \includegraphics[width=0.85\textwidth]{images/论文结构.png} 
    \caption{文章研究框架}
    \label{fig:framework}
\end{figure}


\textbf{第一章:绪论。}首先,介绍研究背景及意义,阐述联邦推荐系统面临的主要问题;其次,介绍联邦推荐的国内外研究现状,按冷启动、数据异质性、通信性能三方面综述,并分析现有研究的不足;最后,说明本文主要研究内容和文章结构安排。

\textbf{第二章:相关理论与技术。}主要对本文研究工作中涉及到的研究领域的基础知识进行介绍,包括推荐系统基础、图神经网络技术、联邦学习框架、对比学习理论等。本章内容是本文研究工作的重要理论基础。

\textbf{第三章:基于属性-结构双视图对比学习的联邦异构图推荐框架。}本章是本文的核心贡献之一,针对联邦推荐系统中的冷启动和数据异质性问题,提出了FedASCL框架。首先,引言部分对本章方案的背景进行介绍以引出研究内容;其次,对所提出的方案技术细节进行详细介绍,包括属性语义图构建、跨视图对比学习、全局语义原型对齐等核心模块;然后,在MovieLens-1M、Yelp、ACM三个数据集上进行实验验证,与5个基线方法进行对比,验证了方法的有效性;最后,对本章进行总结。

\textbf{第四章:基于语义感知的联邦异构图模型参数压缩策略。}本章是本文的另一个核心贡献,针对联邦推荐系统中的通信开销问题,提出语义感知的模型参数压缩策略;在约20倍压缩下保持推荐精度并降低通信开销(见第四章)。

\textbf{第五章:基于联邦推荐模型的推荐系统。}本章用于介绍联邦推荐模型在构建系统方面的实现,包括系统架构设计、核心模块实现、系统功能展示等。本章验证了本文提出的算法在实际工程场景中的可行性。

\textbf{第六章:总结与展望。}本章对全文工作进行总结,阐述本文的主要贡献和创新点,并指出未来可能的研究方向。
