\newpage
\pagenumbering{arabic}\setcounter{page}{1}
% 页眉页脚已在 main.tex 中统一设置,此处不需要重复设置
\pagestyle{fancy}
% \usepackage{indentfirst}  % 已在 option.tex 中加载,此处不需要
\section*{第一章\quad 绪论}
\label{chap:intro}
\setcounter{section}{1} \setcounter{subsection}{0}
\addcontentsline{toc}{section}{第一章\quad 介绍}

% ----------------------------------------------------------------------
% 第一部分:引言
% ----------------------------------------------------------------------
% 本章节将对本文的研究背景进行介绍,引出联邦推荐目前面临的问题与挑战,阐述本文研究工作的意义。本章还会对近年来国内外研究现状进行介绍,并简要阐述本文所做研究工作。最后,给出本文的总体论文框架。

% ----------------------------------------------------------------------
% 第二部分:研究背景
% ----------------------------------------------------------------------
\subsection{研究背景与意义}

随着移动互联网技术的飞速迭代与智能终端设备的全面普及,数字社会产生的信息流呈现爆发式增长。推荐系统(Recommender System)作为解决“信息过载”与实现“用户-内容”精准匹配的核心技术,已深度渗透至电子商务、社交网络及内容聚合平台等关键领域,决定了用户体验的优劣与商业价值的转化效率。

然而,在追求推荐算法精准度的同时,数据隐私安全问题日益凸显。近年来,随着欧盟《通用数据保护条例》(GDPR)\cite{gdpr2016}、中国《中华人民共和国个人信息保护法》(PIPL)\cite{pipl2021}及《数据安全法》等法律法规的相继颁布,数据主权与隐私合规成为不可逾越的红线。在此背景下,联邦学习(Federated Learning, FL)\cite{wu2021fedgnn}作为一种“数据不动模型动”的新型分布式隐私计算范式,被应用在推荐领域。它允许参与方在本地保留原始数据的前提下,仅通过加密交互模型参数或梯度协同构建全局模型,为打破数据孤岛、实现隐私保护下的智能推荐提供了可行的技术路径。

然而,由于联邦学习的特殊性,推荐系统的发展仍然面临冷启动问题、推荐效果变差、通信成本高、落地困难等一系列严峻问题和挑战\cite{ye2023heterogeneous, zhang2024ifedrec}。

这些问题与挑战的内容具体如下:

\begin{enumerate}[wide, labelwidth=!, labelindent=\parindent]
    \item \textbf{数据异质性导致的模型性能瓶颈(Non-IID):} 联邦环境中普遍存在的数据非独立同分布(Non-IID)特性\cite{ye2023heterogeneous}。在传统联邦推荐架构下,各客户端的本地数据分布(如用户交互偏好、物品类别分布)存在显著的统计异质性(Statistical Heterogeneity)。其模型参数(梯度)朝着不同的局部最优方向更新。当服务器对这些“方向冲突”的参数进行聚合时,会引发严重的“客户端漂移”(Client Drift)现象\cite{tan2023fedstar},使得最终生成的全局模型难以收敛到全局最优解,从而导致模型无法精准适配特定用户的个性化需求,整体推荐准确率显著低于集中式训练。
    
    \item \textbf{严重的冷启动问题(Cold Start):} 在传统集中式系统中,新用户即便无交互记录,系统仍可利用其属性特征匹配相似用户群\cite{qiao2018coldstart}。然而在联邦架构下,新用户数据驻留本地,服务器无法利用全局属性信息进行相似度匹配\cite{zhang2024ifedrec}。由于缺乏历史交互数据驱动梯度更新,本地模型因缺乏历史交互数据而难以有效学习,难以通过属性语义弥补拓扑信息的缺失。
    
    \item \textbf{高昂的通信与计算开销(Communication \& Computation Cost):} 异构图推荐模型的核心参数是节点嵌入矩阵,大小随着用户和物品数量线性增长,推荐模型的体积往往高达数百MB\cite{jiang2022fedmp, khan2025hufe},联邦学习通常需要数百轮的迭代才能收敛,会导致网络拥塞,训练速度慢\cite{reisizadeh2020fedpaq}。
    
    \item \textbf{系统工程落地困难(System Deployment):} 目前学术界大多聚焦于算法层面的理论创新,而针对系统架构设计与工程落地的研究却相对匮乏\cite{guo2024fedgnn}。
\end{enumerate}

综上所述,尽管联邦学习为解决推荐系统中的数据孤岛与隐私保护难题提供了极具潜力的技术路径,但上述在数据异质性下的模型性能、冷启动适应能力、系统通信计算开销以及工程化落地等方面存在的严峻挑战,严重制约了其在真实场景中的广泛应用,因此该课题的研究具有重要的理论价值和实践意义。

% ----------------------------------------------------------------------
% 第三部分:现状
% ----------------------------------------------------------------------
% --- 1.2 相关研究现状 ---
\subsection{相关研究现状}

\subsubsection{基于图神经网络的推荐算法研究}
推荐系统作为解决信息过载的核心技术,其发展历程经历了从基于统计学的传统协同过滤(Collaborative Filtering, CF),到基于深度学习的神经协同过滤(Neural CF),再到如今基于图学习的结构化推荐三个阶段。早期的矩阵分解(MF)技术将用户与物品映射为潜在向量,通过内积运算拟合交互得分,但受限于线性组合的表达能力,难以捕捉复杂的非线性特征\cite{hanani2001information}。随着图神经网络(GNN)的兴起\cite{kipf2016semi},研究者发现推荐系统中的“用户-物品”交互数据天然构成了图结构,利用 GNN 在非欧几里得空间中的“嵌入传播(Embedding Propagation)”能力,可以有效捕捉高阶协同信号(High-order Connectivity)\cite{wang2019neural}。根据对图结构建模复杂度的不同,该领域的发展主要经历了基于图的高阶协同传播演进与基于元路径的语义建模两个核心演进阶段。

(1)基于图的高阶协同传播与结构简化:这一阶段的研究核心在于利用图卷积网络(Graph Convolutional Network, GCN)捕捉用户与物品间潜在的高阶协同信号(High-order Connectivity)。
NGCF (Neural Graph Collaborative Filtering)~\cite{wang2019ngcf} 率先奠定了这一领域的基石,它通过“嵌入传播”层显式地模拟了用户偏好的高阶传递过程,有效克服了传统协同过滤(CF)无法利用多跳邻居信息的缺陷。

鉴于仅依赖交互数据的稀疏性局限,KGAT (Knowledge Graph Attention Network)~\cite{wang2019kgat} 进一步将协同传播范式拓展至外部知识图谱(Knowledge Graph, KG)。
该模型利用递归注意力机制聚合实体间的语义关联,不仅增强了物品表征的丰富度,也为推荐结果引入了可解释性。

然而,后续研究表明,NGCF 与 KGAT 中沿用自计算机视觉的非线性激活与特征变换带来了不必要的计算负担。
LightGCN~\cite{he2020lightgcn} 对此进行了理论解构,论证了图推荐中线性聚合的充分性,通过剔除所有非线性操作确立了轻量级图协同过滤的基准。
在此基础上,为追求极致的效率与鲁棒性,UltraGCN~\cite{mao2021ultragcn} 创新性地跳过了显式的消息传递过程,转而通过约束损失函数逼近无限层图卷积,结合谱图理论(Spectral Graph Theory)有效抑制了数据噪声,实现了精度与效率的双重突破。

(2)基于元路径的语义建模: 现实世界的交互数据天然构成了包含多种节点类型与边关系的异构信息网络(HIN)\cite{tan2023nah}。为了从复杂的网络拓扑中精准捕获用户偏好,HAN \cite{wang2019han} 等经典算法通过元路径引导的注意力机制,实现了异构语义的对齐与聚合,证明了利用异构信息辅助推荐精度的有效性。在此基础上,MAGNN (Masked Graph Attention Network) \cite{li2025hgcn} 进一步指出 HAN 忽略了元路径内部中间节点的特征信息,通过引入路径内的编码器与变换器,实现了对长距离语义依赖的更精细化捕捉,从而在序列推荐等任务中提升了预测准确率。

尽管基于元路径的方法在一定程度上提升了性能,但其严重依赖人工先验知识\cite{li2023metapath}。为解决这一局限,HGT (Heterogeneous Graph Transformer) \cite{zhao2024cross} 提出了元关系(Meta-relation)感知的参数化机制,无需人工定义路径即可自动学习异构子图的动态注意力。然而,上述方法仍难以剔除原始交互数据中广泛存在的噪声(如误点击)。为进一步突破推荐精度的瓶颈,HGSL (Heterogeneous Graph Structure Learning) \cite{he2022analyzing} 等算法提出了图结构联合学习的思路。这类方法不再被动地在固定图上进行卷积,而是动态地生成最优的异构图结构,自动识别并强化对推荐目标有益的语义连接,同时抑制噪声干扰。这种端到端的结构优化策略,使得模型能够从纷繁复杂的异构数据中提炼出更纯净的用户意图表征,从而在Top-K推荐任务中取得了更优异的预测表现。

综上所述,基于图神经网络的推荐算法经历了一个从同构图的高阶传播到异构图的语义对齐,再到多行为复杂交互建模的演进过程。算法架构也从盲目堆砌深层网络,逐渐回归到针对推荐任务特性的线性化、语义化和轻量化设计。这为本文在联邦环境下进一步探索异构图的属性-结构双视图对齐奠定了坚实的理论基础。

\subsubsection{图对比学习研究现状分析}
在信息爆炸的数字经济时代,推荐系统作为缓解信息过载的核心引擎,其效能日益受到数据稀疏性与冷启动问题的制约。由于用户交互数据天然呈现长尾分布,基于监督学习的传统推荐范式往往难以从匮乏的反馈中学习到高质量的表征。鉴于此,自监督学习(Self-Supervised Learning, SSL)凭借其从无标签数据中挖掘内生监督信号的能力,逐渐成为学界关注的焦点。特别是图对比学习(Graph Contrastive Learning, GCL),通过最大化不同视图间的互信息来增强节点表示的判别力,已成为提升推荐系统鲁棒性的关键技术路线。

早期的图对比学习探索主要沿袭了计算机视觉中的数据增强思路\cite{oord2018representation},聚焦于图结构的拓扑扰动。以SGL\cite{wu2021sgl}为代表的经典工作,通过随机丢弃边(Edge Dropout)、掩盖节点(Node Masking)或游走子图等方式构建对比视图,旨在利用局部结构的变异性来辅助主任务学习。这种方式原理直观、易于实现,因而在GCL发展的初期得到了广泛应用。然而,图数据具有高度的非欧几里得特性,过度的结构扰动极易破坏图原本的连通性与内在语义,导致“语义漂移”现象。针对这一局限,SimGCL\cite{yu2022simgcl}对此进行了深刻反思,证明了繁复的结构增强并非必要,反而是在嵌入空间注入均匀噪声(Uniform Noise)能够更有效地保持拓扑完整性并提升表征的均匀性(Uniformity)。

随着研究的深入,单纯的随机扰动已难以满足对高阶语义挖掘的需求,研究重心逐渐从“数据增强”向“语义对齐”演进\cite{wang2025fedpcl}。为了捕捉节点在全局视角下的潜在关联,NCL\cite{zhao2024multiinterest}创新性地引入了期望最大化(EM)算法,将节点与其在语义聚类中的原型中心(Cluster Prototypes)进行对比,从而在结构邻居之外捕获了隐式的语义邻居信息。这种基于原型的对比机制有效缓解了结构噪声的影响,但仍主要局限于同构图的范畴。

在现实世界的推荐场景中,用户行为往往蕴含着复杂的异构性与多模态特征\cite{tan2023nah}。为了克服单一交互视图在语义呈现上的局限性,跨视图(Cross-View)对比机制日益得到推崇\cite{wang2025fedpcl}。MACRec\cite{wang2024macrec}与HGCA\cite{he2024hgca}等前沿算法进一步融合了用户属性视图、社交网络视图与交互结构视图,利用互信息最大化原理,强制模型在不同语义空间下保持表征的一致性。例如,通过将用户的行为模式与其静态属性(如年龄、职业)进行对齐,显著增强了模型对冷启动用户的表征能力。

尽管上述方法显著提升了推荐系统的性能,但在对比信号的构建过程中,普遍面临着"假负样本(False Negative)"引入的噪声问题——即原本相似的样本被错误地视作负例排斥。针对此问题,现有研究引入了重加权机制,通过评估样本的可信度来从根本上过滤噪声信号。然而,现有的多视图对比方法在处理复杂异构环境时,往往采用固定权重的多任务学习模式,忽略了不同用户在不同行为模式下的偏好差异。这种缺乏自适应性的对比强度调节,使得模型难以在“推荐主任务”与“对比辅助任务”之间取得最优平衡,这正是本研究旨在解决的核心难点。

\subsubsection{联邦推荐系统研究现状}
联邦学习(Federated Learning)通过在本地保留原始数据的前提下进行分布式
梯度更新,重构了“数据不动模型动”的隐私计算范式\cite{wu2021fedgnn},成为打破推荐系统数据孤岛
的核心路径。早期 FCF \cite{ye2023heterogeneous} 开创性地将矩阵分解扩展至
联邦场景,利用随机梯度下降实现隐私保护下的协同过滤,但其通信开销与单一交
互建模限制了性能,促使 FedFast \cite{reisizadeh2020fedpaq} 通过采样聚合与激活
函数优化确立了高效通信的联邦基准,随后 FedGNN \cite{wu2021fedgnn} 进一步突
破数据本地化对图结构构建的限制,利用隐私保护的用户-物品图扩展策略,实现了
分布式场景下的高阶协同信号捕捉;为解决数据异质性导致的模型偏差\cite{ye2023heterogeneous},
研究重心转向个性化联邦推荐(Personalized FL),FedPerGNN \cite{wu2022fedpergnn}
利用元学习机制实现全局模型对本地异质数据的快速适应,而 FedHGNN \cite{yan2024fedhgnn}
则通过解耦异构图中的私有与共享元路径,实现了异构语义在隐私约束下的
对齐;FedDCSR \cite{zhang2024feddcsr} 通过解耦表示学习实现了跨域联邦推荐。
对比学习作为解决数据稀疏与异质分布的方案被引入联邦推荐\cite{wang2025fedpcl},为缓解 Non-IID
数据造成的全局与局部语义割裂提供了新方式。FCCF \cite{wu2023fccf} 率先提出
基于双重视角的联邦对比框架,通过对齐全局公共表征与本地个性化表征,有效抑制
了噪声梯度的负面影响,PerFedRec++ \cite{luo2024perfedrec} 通过构建多视图协同
对比机制,利用结构扰动生成的辅助视图来校正本地数据的分布偏差,验证了跨视图
增强在联邦环境下的可行性。

随着联邦推荐向移动端、物联网设备等端侧场景延伸,模型轻量化成为解决端侧
算力有限、通信带宽受限的关键方向\cite{jiang2022fedmp},相关研究逐步兴起:FedLiteRec \cite{zhu2023fedlp} 采用结构化剪枝与低秩分解结合的策略,在保留核心交互语义的前提下,将
模型参数量压缩 60\%,同时通过梯度量化降低传输开销;FedQRec \cite{liu2024adaptive} 提出自适应比特量化机制,根据参数重要性动态分配量化比特,核心语义参数采
用 8-bit 高精度量化,冗余参数采用 4-bit 低精度量化,在压缩率达 75\% 的情况下保
持推荐精度损失低于 3\%;FedDistillRec \cite{yi2023fedlpq} 引入联邦知识蒸馏框
架,服务器端训练高精度教师模型,客户端仅部署轻量化学生模型,通过蒸馏损失对
齐师生模型表征,既降低端侧计算负担,又缓解 Non-IID 导致的性能下降。
总体而言,现有研究已初步实现联邦学习与图神经网络的结合\cite{wu2021fedgnn},并在同构图对比
增强方面取得进展\cite{wu2021sgl},但在联邦隐私约束下,针对异构信息网络(HIN)复杂语义(如
元路径依赖、属性-结构跨模态交互)的对比学习机制仍显匮乏\cite{tan2023nah},在模型轻量化方面,还
未发现基于元路径的剪枝优化方面的研究。


\subsubsection{现有研究总结与不足}
综上所述,图神经网络、图对比学习与联邦学习的结合在提升推荐精度与保护用户隐私方面展现了巨大潜力,但深度融合过程中仍存在诸多亟待解决的问题。首先,由于联邦学习的物理隔离性,单个客户端仅持有局部交互子图,导致现有的对比机制难以捕获跨域的长程拓扑关联,存在全局结构感知受限的缺陷。其次,针对异构场景下的多行为数据,如何在保护复杂语义隐私的同时实现高效的自适应对比增强,目前的方案仍显匮乏。最后,在隐私噪声与通信压缩的双重约束下,如何保持对比学习生成的表征具有足够的辨别力,依然是制约联邦图对比推荐系统落地的核心矛盾。本研究旨在针对上述不足,提出一种融合异构语义感知与高效通信机制的联邦图对比推荐算法,以期在隐私合规的前提下实现高性能的个性化推荐。
% ----------------------------------------------------------------------
% 第四部分:贡献
% ----------------------------------------------------------------------
\subsection{本文研究工作和主要贡献}

本文研究旨在探讨联邦推荐系统面临的若干关键问题,包括冷启动问题、推荐效果差、通信成本高以及系统实现困难。为了应对这些问题与挑战,本文主要研究包括如下内容:第三章提出 FedASCL 框架解决冷启动和数据异质性问题;第四章提出语义感知压缩策略解决通信开销问题;第五章设计并实现联邦推荐系统解决系统落地问题。

\begin{enumerate}[wide, labelwidth=!, labelindent=\parindent]
    \item 针对冷启动和非独立同分布导致的推荐效果差的核心问题,本文提出了\textbf{基于属性-结构双视图对比学习的联邦异构图推荐框架(FedASCL)}。该框架通过跨视图对比学习机制,在潜在空间对齐用户属性与图结构语义;利用属性特征重构缺失的拓扑信息,有效解决了零交互用户的冷启动难题。同时,针对数据异质性导致的模型偏差问题,该框架引入了全局语义原型对齐机制,通过约束本地模型表征向全局一致的语义原型靠拢,有效纠正了客户端模型漂移现象,从而显著缓解了推荐性能的下降。
    
    \item 针对联邦异构图模型参数量大、通信效率低的问题,本文提出了语义感知的模型参数压缩策略。该策略摒弃了传统的粗粒度压缩方法,设计了一个轻量级的动态元路径选择器,利用可学习的注意力权重在本地训练阶段自动识别并剔除对当前用户意图贡献度低的冗余语义通道,从结构上实现模型剪枝;在此基础上,进一步引入残差梯度量化技术\cite{alistarh2017qsgd},将剩余的关键参数梯度压缩为低比特整数进行传输,并将量化过程中产生的精度误差累积存储在本地以在下一轮更新中进行补偿;这种"先剪枝后量化"的组合策略在保证模型语义完整性与收敛精度的前提下,成功将通信开销降低了一个数量级。
    
    \item 针对当前联邦推荐系统落地困难的问题,本文构建了支持异构设备协同的联邦论文推荐系统。该系统采用端云协同的异步聚合架构以适配不同性能的终端设备,并集成了本地差分隐私模块,完成了联邦推荐系统的实际落地验证。
\end{enumerate}

% ----------------------------------------------------------------------
% 第五部分:结构
% ----------------------------------------------------------------------
\subsection{论文结构}

文章共六章,分别涉及本文研究的背景、意义、相关研究基础、算法模型、系统实现以及总结和展望。文章研究框架如图 \ref{fig:framework} 所示。

\begin{figure}[H]
    \centering
    \includegraphics[width=0.85\textwidth]{images/论文结构.png} 
    \caption{文章研究框架}
    \label{fig:framework}
\end{figure}


\textbf{第一章:绪论。}首先,介绍研究背景及意义。其次,介绍联邦推荐的国内外研究现状。最后,说明本文主要研究内容和文章结构安排。

\textbf{第二章:相关理论与技术。}主要对本文研究工作中涉及到的研究领域的基础知识进行介绍。本章内容是本文研究工作的重要基础。

\textbf{第三章:基于属性-结构双视图对比学习的联邦异构图推荐框架。}首先,引言部分对本章方案的背景进行介绍以引出研究内容。其次,对所提出的方案技术细节进行详细介绍。然后,对实验结果进行分析。最后,对本章进行总结。

\textbf{第四章:基于语义感知的联邦异构图模型参数压缩策略。}本章结构安排与第三章相似,同样分为引言部分、方案部分、实验部分以及总结部分。

\textbf{第五章:基于联邦推荐模型的推荐系统。}本章用于介绍联邦推荐模型在构建系统方面的实现。

\textbf{第六章:总结与展望。}
