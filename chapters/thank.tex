% chapters/thank.tex

% --- 标题设置 ---
\newpage
\section*{致\quad 谢}
\label{sec:thank}
\addcontentsline{toc}{section}{致谢}

% --- 正文设置 ---
\songti \zihao{-4} % 设置正文为宋体、小四号
\setlength{\parindent}{2em}     % 首行缩进2字符
\setlength{\baselineskip}{22pt} % 行间距固定为22pt(约1.5倍)

行文至此,落笔为终。随着毕业论文最后一个句号的画下,我的硕士研究生生涯也即将落下帷幕。回首在西北大学度过的这三年时光,仿佛就在昨日。长安校区的玉兰花开花落,图书馆深夜长明的灯光,以及无数个在实验室埋首攻关的日夜,都化作了此刻心头最温柔的牵挂。求索之路虽有艰辛,有过迷茫与焦虑,但沉淀下来的,更多的是沉甸甸的收获与成长。

首先,我要向我的恩师吴昊老师致以最崇高的敬意和最深切的感谢。本论文从选题的确定、研究方案的设计到最终的定稿,每一步都离不开吴老师的悉心指导。在每一次的组会讨论中,吴老师不仅教会了我如何发现问题、解决问题,更以身作则地教会了我作为一名科研工作者应有的严谨与责任,吴老师的谆谆教诲将是我未来人生道路上宝贵的财富。

特别感谢实验室的同窗好友们,这段并肩作战、相互扶持的情谊,是我研究生阶段最美好的回忆,我将永远铭记于心。

感谢我的父母和家人。二十余载求学路,是你们用无私的爱与包容,为我筑起了最坚实的后盾,你们永远是我最忠实的听众和最温暖的港湾。感谢你们多年来默默的付出与支持,你们的理解与关爱是我克服困难、不断前行的动力源泉。

感谢所有参考文献中的学者前辈,你们的研究成果是本文写作的基石,指引我在巨人的肩膀上眺望更远的风景。

最后,也要感谢那个从未放弃的自己。感谢自己在无数个想退缩的时刻选择了坚持,在无数次失败后选择了重头再来。凡是过往,皆为序章。

路漫漫其修远兮,吾将上下而求索。毕业并非终点,而是新征程的起点。我将带着在西北大学学到的知识与精神,铭记"公诚勤朴"的校训,怀揣热忱,脚踏实地,奔赴山海,去迎接未来的挑战与机遇。

% --- 落款部分 ---
\vspace{2cm} % 距离正文的垂直间距

\begin{flushright}
    作\quad 者:伍勋高 \\
    2026年3月于西北大学
\end{flushright}