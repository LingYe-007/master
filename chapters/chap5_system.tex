\newpage
% =========================================================
% 第五章:基于联邦异构图模型的推荐系统设计与实现
% =========================================================

\section*{第五章\quad 基于联邦异构图模型的推荐系统设计与实现}
\setcounter{section}{5} 
\setcounter{subsection}{0}
\addcontentsline{toc}{section}{第五章\quad 基于联邦异构图模型的推荐系统设计与实现}
\label{chap:system_design}

\subsection{系统概述}
\label{sec:sys_overview}

基于前文提出的 FedASCL(Federated Attribute-Structure Contrastive Learning)算法框架,本章设计并实现了一个面向学术场景的隐私保护联邦推荐系统。系统用于解决传统集中式推荐中存在的"数据孤岛"与隐私泄露问题。通过构建"端-云协同"的联邦计算架构,系统能够在客户端数据不出域的同时,利用本地自我中心子图(Ego-Subgraph)与全局语义原型协同训练,完成高精度的个性化推荐。

\subsection{系统设计原则}
为了兼顾推荐系统的准确性、实时性与安全性,系统的设计遵循以下核心原则:

\begin{enumerate}
    \item \textbf{数据隐私优先(Data Privacy First):} 严格遵循联邦学习协议,原始交互数据(如点击流)与敏感属性(如所属机构)仅存储于客户端本地。端云之间仅交互加密后的模型梯度或原型参数,从架构层面阻断隐私泄露风险。
    \item \textbf{架构分层解耦(Hierarchical Decoupling):} 采用微服务架构思想,将核心算法逻辑(联邦训练、梯度聚合)与业务应用逻辑(推荐展示、用户管理)分离,降低系统耦合度,便于算法模型的独立迭代。
    \item \textbf{异构数据适配(Heterogeneity Adaptation):} 针对学术场景中节点类型多样(论文、作者、会议)的特点,系统在端侧设计了动态元路径编码器,能够自适应处理不同构型的本地子图数据。
\end{enumerate}

\subsection{系统总体架构设计}
系统采用分层架构设计,自下而上划分为数据存储层、算法引擎层、业务服务层、业务展示层,如图 \ref{fig:sys_arch} 所示。

\begin{figure}[htbp]
    \centering
    \rule{0.8\textwidth}{0.4pt}\\[1em]
    \centering
    \textbf{[待插入图片:系统总体架构图]}\\[1em]
    \rule{0.8\textwidth}{0.4pt}
    \caption{基于 FedASCL 的联邦推荐系统逻辑架构图}
    \label{fig:sys_arch}
\end{figure}

各层级的功能定义如下:

\begin{enumerate}
    \item \textbf{数据存储层 (Data Storage Layer):}
    数据存储层采用客户端-服务端分离的架构设计。客户端存储结构示意图展示了用户子图的本地存储方式,体现了联邦学习中数据不出域的核心原则。如图 \ref{fig:storage_layer} 所示,每个客户端仅存储以当前用户为中心的子图结构,包括用户节点及其直接交互的论文、作者等节点,有效保护了用户隐私。
    
    \begin{figure}[htbp]
        \centering
        \includegraphics[width=0.95\textwidth]{images/用户子图示意图.png}
        \caption{客户端用户子图存储示意图}
        \label{fig:storage_layer}
    \end{figure}
    
    数据存储层的具体实现包括:
    \begin{enumerate}
        \item \textbf{客户端(Client):} 部署于用户终端。利用轻量级数据库(如 SQLite)存储本地私有的自我中心子图(Ego-Subgraph)及用户属性特征。
        \item \textbf{服务端(Cloud):} 利用 MySQL 存储系统元数据(用户信息、日志),利用 Redis 缓存待分发的全局模型参数(Global Model)及全局语义原型(Global Prototypes)。
    \end{enumerate}
    
    \item \textbf{算法引擎层 (Algorithm Engine Layer):}
    该层是系统的核心计算单元,封装了 FedASCL 框架的关键组件。算法引擎层的主要组件包括:
    \begin{enumerate}
        \item \textbf{本地训练器 (Local Trainer):} 负责在客户端执行基于属性-结构双视图的对比学习,计算 InfoNCE 损失与原型对齐损失。
        \item \textbf{联邦协调器 (Federated Coordinator):} 部署于云端,负责任务调度、节点选择以及全局原型的同步与更新。
        \item \textbf{安全聚合器 (Secure Aggregator):} 执行基于加权平均(FedAvg)的参数聚合,融合各客户端上传的梯度信息。
    \end{enumerate}
    
    \item \textbf{业务服务层 (Service Layer):}
    基于算法层的输出提供具体的 API 服务,包括推荐列表生成服务(Recommendation Service)和新用户冷启动处理服务(Cold-start Handler)。
    
    \item \textbf{业务展示层 (Presentation Layer):}
    提供可视化的 Web 控制台,支持学术文献浏览以及联邦训练状态的实时监控(如 Loss 曲线、AUC 指标)。
\end{enumerate}

\subsection{技术选型与开发环境}
系统基于 Python 生态构建,采用前后端分离模式。核心技术选型如表 \ref{tab:tech_stack} 所示。

\begin{table}[htbp]
    \centering
    \caption{系统技术选型与开发环境}
    \label{tab:tech_stack}
    \renewcommand{\arraystretch}{1.1}
    \begin{tabular}{l|l}
        \toprule
        \textbf{组件类别} & \textbf{技术选型} \\
        \midrule
        \textbf{开发语言} & Python 3.8 \\
        \textbf{深度学习框架} & PyTorch 1.10, PyTorch Geometric (PyG) \\
        \textbf{通信协议} & RESTful API / gRPC (模型参数传输) \\
        \textbf{后端框架} & Flask (API服务), Celery (异步任务调度) \\
        \textbf{数据存储} & MySQL 8.0 (元数据), Redis 6.0 (热数据) \\
        \textbf{前端框架} & React + TypeScript + ECharts (可视化) \\
        \bottomrule
    \end{tabular}
\end{table}

\subsection{关键模块详细设计}
\label{sec:sys_modules}

\subsubsection{联邦协同训练模块}
该模块是系统智能化的核心,负责协调分布式的客户端与中心服务器进行多轮次的联邦学习。

\paragraph{1. 训练流程控制}
系统采用同步联邦学习(Synchronous FL)机制,单轮训练主要包含四个标准阶段:

\begin{enumerate}
    \item \textbf{模型分发(Broadcast):} 服务端根据节点活跃度,随机选择 $K$ 个客户端,下发最新的全局模型参数 $W_g$ 与全局属性原型 $P_g$。
    \item \textbf{本地更新(Local Update):} 客户端加载模型,基于本地异构子图执行 $E$ 轮 FedASCL 训练。在此过程中,双视图编码器分别提取结构特征与属性特征,并通过最大化互信息来优化节点表示。
    \item \textbf{参数上传(Upload):} 训练结束后,客户端仅上传模型梯度更新 $\Delta W$ 与本地属性聚类中心,原始图数据严格保留在本地。
    \item \textbf{全局聚合(Aggregation):} 服务端收集所有被选节点的更新,执行加权聚合操作,更新全局模型版本。
\end{enumerate}

\paragraph{2. 仿真引擎实现}
为了验证算法有效性,系统后端实现了一个高保真仿真引擎。引擎通过 \texttt{DataPartitioner} 模块将 ACM 数据集切分为多个互不重叠的 Ego-Subgraph,模拟真实环境下的数据异构(Non-IID)特性;通过多线程机制模拟客户端的并发训练与网络延迟。仿真引擎的架构设计如图 \ref{fig:simulator_arch} 所示。

\begin{figure}[htbp]
    \centering
    \includegraphics[width=0.9\textwidth]{images/联邦学习仿真引擎架构图.png}
    \caption{联邦学习仿真引擎架构图}
    \label{fig:simulator_arch}
\end{figure}

\subsubsection{学术资源推荐模块}
\label{subsec:paper_rec}
该模块基于训练好的联邦异构图模型,为用户提供个性化的学术资源推荐,重点解决了冷启动问题。

\paragraph{1. 分级推荐策略}
针对学术场景中用户交互稀疏的特点,该功能实现了"分级推理"机制,其逻辑判定流程如图 \ref{fig:rec_logic} 所示:

\begin{figure}[htbp]
    \centering
    \includegraphics[width=0.9\textwidth]{images/分级推荐策略.jpg}
    \caption{基于属性-结构双通路的冷启动推荐策略示意图}
    \label{fig:rec_logic}
\end{figure}

\begin{enumerate}
    \item \textbf{常规推荐(结构优先):} 对于有丰富交互历史的活跃用户,系统优先利用结构视图,聚合"用户-论文-会议"的高阶邻居信息,生成精确的用户嵌入,检索 Top-N 相似文献。
    \item \textbf{冷启动推荐(属性优先):} 对于无交互记录的新注册用户,系统自动激活 FedASCL 的\textbf{属性映射分支}。利用用户注册时的属性(如研究兴趣、机构),通过映射网络投影到全局原型空间,基于语义相似度检索相关文献,实现零样本启动。
\end{enumerate}

\paragraph{2. 可解释性增强}
系统利用动态元路径的注意力权重(Attention Weights)提供推荐解释。例如,若 \texttt{User-Author-User} 路径的权重较高,前端界面将提示用户:"推荐该学者是因为你们有共同的合作者"。

\subsection{系统界面展示与功能测试}
\label{sec:sys_ui_test}

系统界面展示部分包括学术资源推荐、联邦训练监控以及用户个人设置三个主要模块。

\paragraph{1. 学术资源推荐列表页}
学术资源推荐列表页是系统的核心功能界面,展示了系统为用户推荐的个性化学术论文列表。该界面包含论文标题、作者、发表会议等关键信息,帮助用户快速了解推荐论文的基本信息。同时,系统还提供了基于元路径的推荐解释功能,用户可以通过查看元路径注意力权重了解推荐理由,例如"推荐该论文是因为您与作者有共同的合作者"或"该论文与您的研究兴趣高度相关"。学术资源推荐列表页如图 \ref{fig:ui_rec} 所示。

\begin{figure}[htbp]
    \centering
    \includegraphics[width=0.9\textwidth]{images/论文推荐页面.png}
    \caption{学术资源推荐列表页}
    \label{fig:ui_rec}
\end{figure}

\paragraph{2. 联邦训练状态监控面板}
联邦训练状态监控面板是中央服务器端的重要管理工具,为系统管理员提供了全面的联邦学习训练监控功能。该界面实时展示参与训练的客户端数量、当前训练轮次、全局模型性能指标(Loss、AUC等)的实时曲线变化,以及各客户端的训练状态(在线/离线、训练进度等)。通过该监控面板,管理员可以及时发现训练异常、评估模型收敛情况,并做出相应的调度决策。联邦训练状态监控面板如图 \ref{fig:ui_monitor} 所示。

\begin{figure}[htbp]
    \centering
    \includegraphics[width=0.9\textwidth]{images/中央服务器训练面板.png}
    \caption{联邦训练状态监控面板}
    \label{fig:ui_monitor}
\end{figure}

\paragraph{3. 用户个人设置页面}
用户个人设置页面是系统冷启动推荐功能的重要支撑模块。该界面允许用户设置个人研究兴趣、所属机构、研究方向等属性信息。这些属性信息作为 FedASCL 算法中属性视图的核心数据源,在用户缺乏历史交互记录时,系统能够基于属性语义图构建用户表示,实现零样本启动。同时,所有属性数据严格存储在客户端本地,仅在上传模型参数时以加密形式参与联邦训练,充分体现了系统对用户隐私数据的本地化管理原则。用户个人设置页面如图 \ref{fig:ui_settings} 所示。

\begin{figure}[htbp]
    \centering
    \includegraphics[width=0.9\textwidth]{images/个人设置页面.png}
    \caption{用户个人设置页面}
    \label{fig:ui_settings}
\end{figure}

\paragraph{系统功能测试结果}
测试采用黑盒测试方法,重点验证系统各核心功能模块的逻辑正确性与业务流程的闭环能力。测试环境覆盖了模拟的客户端节点与中心服务器,测试结果如表 \ref{tab:func_test} 所示。

\begin{table}[htbp]
    \centering
    \caption{系统核心功能测试用例及结果}
    \label{tab:func_test}
    \renewcommand{\arraystretch}{1.2}
    \resizebox{\linewidth}{!}{
    \begin{tabular}{c|p{6cm}|p{4cm}|c}
        \toprule
        \textbf{测试模块} & \textbf{测试用例描述} & \textbf{预期结果} & \textbf{结论} \\
        \midrule
        \multirow{2}{*}{\textbf{联邦控制}} & 验证全局模型参数的版本迭代 & 轮次 $t+1$ 的参数与 $t$ 轮发生更新 & 通过 \\
        \cline{2-4}
         & 验证数据隔离性 & 服务端无法读取客户端的原始图结构 & 通过 \\
        \midrule
        \multirow{2}{*}{\textbf{推荐服务}} & 历史交互丰富的老用户请求推荐 & 推荐列表与历史偏好领域一致性高 & 通过 \\
        \cline{2-4}
         & 仅有属性信息的冷启动用户请求推荐 & 系统基于属性原型返回相关领域文献 & 通过 \\
        \midrule
        \textbf{系统监控} & 前端实时拉取训练 Loss 与 AUC 指标 & 曲线随训练轮次动态更新且趋势收敛 & 通过 \\
        \bottomrule
    \end{tabular}}
\end{table}

测试结果表明,系统成功实现了联邦学习的完整闭环,具备了模型分发、本地训练、梯度聚合以及基于模型的在线推荐功能。特别是在隐私保护方面,系统架构有效地隔离了原始数据,验证了 FedASCL 框架在工程落地上的可行性。

\subsection{本章小结}
\label{sec:sys_summary}

本章详细阐述了基于 FedASCL 算法的联邦推荐系统的设计与实现过程。首先,确立了端云协同的分层系统架构,明确了数据流向与隐私边界;其次,重点介绍了联邦协同训练与学术资源推荐两大核心模块的实现逻辑,特别是将冷启动处理与动态元路径机制转化为具体的工程特性;最后,通过系统界面展示与功能测试,验证了系统在功能完备性、推荐准确性以及隐私保护方面的优势。该系统的实现为异构图联邦学习在实际学术推荐场景中的应用提供了有效的工程范例。