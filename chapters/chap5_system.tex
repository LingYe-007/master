\newpage
\section*{第五章~~基于联邦异构图模型的推荐系统设计与实现}
\setcounter{section}{5} \setcounter{subsection}{0}
\addcontentsline{toc}{section}{第五章~~基于联邦异构图模型的推荐系统设计与实现}
\label{chap:system_design}

\subsection{系统概述与架构设计}
\label{sec:sys_overview}

基于前文提出的 FedASCL 算法框架与语义感知压缩策略,本章设计并实现了一个面向学术场景的隐私保护联邦推荐系统原型。该系统旨在解决传统集中式推荐中存在的“数据孤岛”与隐私泄露问题,通过端云协同(Cloud-Edge Collaboration)架构,使分布在不同科研机构(客户端)的异构数据能够在不离本地的前提下,协同训练出高精度的异构图推荐模型。

\subsection{系统设计原则}
系统的设计遵循以下核心原则:
\begin{itemize}
    \item \textbf{隐私优先(Privacy-First)}:原始交互数据(如用户阅读记录)与敏感属性6y y y y y(如所属课题组)严禁出域,仅交互加密后的梯度或模型参数。
    \item \textbf{高效通信(Communication-Efficient)}:针对移动端或边缘网络环境,集成第4章提出的语义压缩与残差量化机制,降低传输开销。
    \item \textbf{模块解耦(Decoupling)}:将算法核心逻辑(训练、聚合)与业务逻辑(推荐展示、用户管理)分离,便于后续算法迭代。
\end{itemize}

\subsection{系统分层架构}
系统采用分层微服务架构设计,自下而上划分为数据层、算法引擎层、服务层与应用层,如图 \ref{fig:sys_arch} 所示。



\begin{enumerate}
    \item \textbf{数据存储层 (Data Layer)}:
    \begin{itemize}
        \item \textbf{客户端(Edge)}:部署于用户终端或机构服务器,利用 SQLite/LevelDB 存储本地私有的用户-物品交互图、用户属性特征及本地训练产生的残差累积量。
        \item \textbf{服务端(Cloud)}:利用 MySQL 存储系统元数据(用户信息索引、版本控制),利用 Redis 缓存待分发的全局模型参数及全局原型聚类中心(Global Prototypes)。
    \end{itemize}
    
    \item \textbf{算法引擎层 (Algorithm Engine Layer)}:
    这是系统的核心计算单元,封装了 FedASCL 框架的关键组件。
    \begin{itemize}
        \item \textbf{本地训练器 (Local Trainer)}:负责执行基于属性-结构双视图的对比学习,生成本地梯度。
        \item \textbf{语义压缩器 (Semantic Compressor)}:集成“动态元路径选择器”,在上传前对梯度进行结构剪枝与残差量化。
        \item \textbf{安全聚合器 (Secure Aggregator)}:部署于云端,执行基于加权平均(FedAvg)的参数聚合与原型对齐(Prototype Alignment)。
    \end{itemize}
    
    \item \textbf{服务层 (Service Layer)}:
    基于算法层的输出提供具体的 API 服务,包括 `Recommendation Service`(推荐列表生成)、`Cold-start Handler`(冷启动处理)以及 `Model Version Manager`(模型版本管理)。
    
    \item \textbf{应用层 (Application Layer)}:
    提供可视化的 Web 交互界面,支持学术文献浏览、学者合作网络探索以及联邦训练状态监控面板。
\end{enumerate}

\subsection{技术选型与开发环境}
系统基于 Python 生态构建,核心技术栈如表 \ref{tab:tech_stack} 所示。

\begin{table}[htbp]
    \centering
    \caption{系统技术选型与开发环境}
    \label{tab:tech_stack}
    \begin{tabular}{l|l}
        \toprule
        \textbf{组件类别} & \textbf{技术选型} \\
        \midrule
        \textbf{开发语言} & Python 3.8 \\
        \textbf{深度学习框架} & PyTorch 1.10, PyTorch Geometric (PyG) \\
        \textbf{联邦通信协议} & gRPC / WebSocket (自定义压缩协议报文) \\
        \textbf{后端框架} & Flask (API服务), Celery (异步任务) \\
        \textbf{数据存储} & MySQL 8.0 (关系型), Redis 6.0 (缓存), SQLite (端侧) \\
        \textbf{前端框架} & Vue.js + ECharts (可视化) \\
        \bottomrule
    \end{tabular}
\end{table}

\subsection{关键模块详细设计}
\label{sec:sys_modules}

\subsection{联邦协同训练模块}
该模块负责协调客户端与服务端进行多轮次的联邦学习,是系统智能化的基础。

\subsubsection{通信协议设计}
为了支持第4章提出的压缩策略,自定义了联邦通信报文格式。报文负载(Payload)不再是完整的权重矩阵,而是包含以下字段:
\begin{itemize}
    \item \texttt{meta\_path\_mask}: 二进制掩码,标识当前轮次保留的活跃元路径通道。
    \item \texttt{quantized\_grads}: 经过 INT8 量化的梯度数据流。
    \item \texttt{scalar\_factor}: 用于反量化的缩放因子。
\end{itemize}

\subsubsection{训练流程控制}
系统采用同步联邦学习(Synchronous FL)机制:
\begin{enumerate}
    \item \textbf{分发阶段}:服务端通过心跳包监测在线客户端,随机选择 $K$ 个节点下发最新的全局模型 $W_g$ 与全局原型 $P_g$。
    \item \textbf{本地更新}:客户端加载模型,利用本地异构图数据进行 $E$ 轮 FedASCL 训练。在此过程中,双视图对比学习模块计算 InfoNCE 损失,优化节点表示。
    \item \textbf{压缩上传}:训练结束后,调用`Compressor`模块计算梯度,剔除低注意力权重的元路径分支,并将剩余参数量化打包上传。
    \item \textbf{聚合更新}:服务端等待所有被选节点回传完毕,解压数据并执行加权聚合,更新全局模型版本。
\end{enumerate}



\subsection{学术资源推荐模块}
\label{subsec:paper_rec}
该模块基于训练好的联邦异构图模型,为用户提供个性化的学术资源推荐,主要包括文献推荐与学者合作推荐。

\subsubsection{文献推荐与冷启动处理}
针对学术场景中用户交互稀疏的特点,该功能实现了“分级推理”策略:
\begin{itemize}
    \item \textbf{常规推荐}:对于活跃用户,系统加载本地保存的异构图结构,利用 HGNN 编码器聚合“用户-点击-论文”及“论文-引用-论文”的高阶邻居信息,生成用户嵌入向量,并在候选集中检索 Top-N 相似文献。
    \item \textbf{冷启动推荐}:对于无交互记录的新注册用户,系统自动切换至 \textbf{FedASCL 属性视图}分支。利用用户注册时填写的“研究兴趣关键词”或“所属实验室”属性,映射到全局原型空间,检索与该原型最接近的文献簇,实现零样本启动(Zero-shot Start)。
\end{itemize}

\subsubsection{学者合作网络挖掘}
该功能旨在解决学术社交中的信息不对称问题。区别于文献推荐,学者推荐更关注“同质性”与“互补性”。
系统利用动态元路径机制,在推理阶段动态调整注意力权重,重点关注 \texttt{User-Paper-User} (共同作者) 和 \texttt{User-Org-User} (同一机构) 这类社交元路径。前端界面除了展示推荐的学者列表外,还会生成\textbf{“推荐解释路径”}(例如:“因为你们都引用了 Bengio 的文献”),增强系统的可解释性。

\subsection{系统测试与评估}
\label{sec:sys_test}

\subsection{功能测试}
测试采用黑盒测试方法,验证系统各核心功能模块的逻辑正确性。测试环境覆盖了 Windows 与 Linux 客户端,测试结果如表 \ref{tab:func_test} 所示。

\begin{table}[htbp]
    \centering
    \caption{系统核心功能测试用例及结果}
    \label{tab:func_test}
    \begin{tabular}{c|p{6cm}|c|c}
        \toprule
        \textbf{模块} & \textbf{测试用例描述} & \textbf{预期结果} & \textbf{结论} \\
        \midrule
        \multirow{2}{*}{联邦控制} & 模拟客户端网络波动导致的中断重连 & 服务端剔除超时节点,系统不崩溃 & 通过 \\
        & 验证全局模型参数的版本迭代 & 轮次 $t+1$ 的参数与 $t$ 轮不同 & 通过 \\
        \midrule
        \multirow{2}{*}{推荐服务} & 历史交互丰富的老用户请求 & 推荐列表与历史偏好一致性高 & 通过 \\
        & 仅有属性信息的冷启动用户请求 & 基于属性原型返回相关领域文献 & 通过 \\
        \midrule
        \multirow{2}{*}{隐私保护} & 尝试在服务端解析客户端上传的数据包 & 仅能解析出梯度噪声,无法还原交互图 & 通过 \\
        \bottomrule
    \end{tabular}
\end{table}

\subsection{通信性能测试}
为了验证第4章所提压缩算法在工程实现层面的收益,我们在模拟的弱网环境(带宽限制 5Mbps,延迟 200ms)下进行了压力测试。

\begin{figure}[htbp]
    \centering
    % 
    % \includegraphics[width=0.7\textwidth]{images/latency_test.png} 
    \caption{不同网络环境下单轮联邦通信耗时对比}
    \label{fig:latency_test}
\end{figure}

测试指标包括单轮通信耗时(Communication Latency)和吞吐量。实验结果显示:
\begin{enumerate}
    \item \textbf{时延降低}:在开启语义感知压缩策略后,单轮通信的平均耗时从 4.5 秒降低至 0.6 秒,降幅达到 86\%。这主要得益于动态元路径剪枝大幅减少了无效参数的传输。
    \item \textbf{稳定性提升}:在丢包率设置为 5\% 的不稳定网络下,全精度模型的上传成功率仅为 82\%,而压缩后的模型上传成功率达到 98\%,证明了轻量化模型在边缘环境下的传输鲁棒性。
\end{enumerate}

\subsection{本章小结}
\label{sec:sys_summary}
本章详细阐述了基于 FedASCL 算法的联邦推荐系统的设计与实现过程。
首先,确立了端云协同的四层系统架构,明确了数据流向与隐私边界;
其次,重点介绍了联邦协同训练与学术资源推荐两大核心模块的实现逻辑,特别是将冷启动处理与动态元路径机制转化为具体的工程特性;
最后,通过系统功能与性能测试,验证了系统在功能完备性、推荐准确性以及通信效率方面的优势。该系统的成功实现,标志着本文提出的理论算法具备在实际移动互联网环境下部署的可行性。