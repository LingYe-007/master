\newpage
% =========================================================
% 第五章:基于联邦异构图模型的推荐系统设计与实现
% =========================================================

\section{第五章\quad 基于联邦异构图模型的推荐系统设计与实现}
\label{chap:system_design}
\setcounter{section}{5}
\setcounter{subsection}{0}

本章将第三章提出的 FedASCL 算法框架与第四章的语义感知压缩策略集成到一个完整的学术论文推荐系统中。该系统用于解决传统集中式推荐中的隐私泄露风险,并应对实际落地中面临的数据异质性、通信瓶颈等工程问题。本章按照系统开发的标准流程进行介绍,主要分为系统需求分析、系统总体架构设计、关键模块实现以及系统功能测试四个部分。

% ----------------------------------------------------------------------
\subsection{系统需求分析}
\label{sec:sys_requirements}

本节从业务逻辑与用户操作两个维度对联邦推荐系统进行功能定义。

\subsubsection{业务需求分析}

为了构建一个安全、高效且可扩展的学术推荐平台,系统需满足以下业务流程:

\textbf{(1)联邦身份认证:} 支持用户注册并建立本地私有数据库,确保原始交互数据不出域。端云之间仅交互加密后的模型梯度或原型参数,从架构层面阻断隐私泄露风险。

\textbf{(2)端云协同训练:} 支持服务端下发全局原型与模型参数,客户端执行本地双视图对比学习(属性视图与结构视图),并完成梯度的压缩上传;服务端执行加权聚合,更新全局模型版本。

\textbf{(3)隐私安全保障:} 集成本地差分隐私模块,对上传的原型或梯度添加噪声,防御恶意攻击与标签推理攻击。

\textbf{(4)高效通信管理:} 应用动态元路径剪枝与残差梯度量化(第四章语义感知压缩策略),在带宽受限环境下完成模型迭代,降低通信开销。

上述业务功能需求的关系与数据流如图~\ref{fig:sys_arch} 所示(系统总体设计架构图见 5.2.1 节)。

\subsubsection{用户需求分析}

系统主要面向学术研究人员,主要功能模块包括:

\textbf{(1)个人画像管理模块:} 用户可自主维护研究兴趣、所属机构、研究方向等属性,作为冷启动阶段属性视图的输入;所有属性数据严格存储在客户端本地,仅以加密形式参与联邦训练。

\textbf{(2)联邦训练监控模块:} 为管理员提供实时训练状态看板,包括 Loss 曲线、AUC 指标及客户端在线状态,便于发现训练异常、评估模型收敛情况并做出调度决策。

\textbf{(3)学术资源推荐模块:} 根据用户是否有交互历史,自动切换“结构优先”或“属性优先”的推荐策略,展示个性化论文列表,并支持基于元路径注意力权重的推荐理由展示。

系统用户需求与上述模块的对应关系已在 5.3.3 与 5.4 节的实现与测试中体现。

% ----------------------------------------------------------------------
\subsection{系统总体架构设计}
\label{sec:sys_architecture}

本系统采用分层架构设计,实现了算法逻辑与业务应用的高度解耦。

\subsubsection{逻辑架构设计}

系统自下而上划分为四层,如图~\ref{fig:sys_arch} 所示。

\textbf{(1)数据存储层:} 实现客户端本地 SQLite(存储自我中心子图及用户属性)与服务端 MySQL/Redis(存储全局参数与元数据)的分离,满足数据不出域原则。

\textbf{(2)算法引擎层:} 封装 FedASCL 主要算法(双视图对比学习、全局语义原型对齐)、语义感知压缩器(动态元路径选择器、残差梯度量化)及安全聚合组件(FedAvg)。

\textbf{(3)业务服务层:} 提供推荐列表生成、冷启动处理及异步任务调度等 API 服务。

\textbf{(4)业务展示层:} 基于 React 构建的 Web 交互界面,支持学术文献浏览与联邦训练状态监控。

\begin{figure}[H]
    \centering
    \includegraphics[width=0.85\textwidth]{images/system_architecture_cn.png}
    \caption{系统总体设计架构图(自顶向下)}
    \label{fig:sys_arch}
\end{figure}

\subsubsection{数据不出域机制设计}

数据存储层采用客户端-服务端分离的架构。每个客户端仅存储以当前用户为中心的自我中心子图(Ego-Subgraph),包括用户节点及其直接交互的论文、作者等节点,原始图数据与敏感属性严格保留在本地。本地子图存储方式如图~\ref{fig:storage_layer} 所示。

\begin{figure}[H]
    \centering
    \includegraphics[width=0.95\textwidth]{images/用户子图示意图.png}
    \caption{客户端用户子图存储示意图}
    \label{fig:storage_layer}
\end{figure}

% ----------------------------------------------------------------------
\subsection{系统关键模块实现}
\label{sec:sys_modules}

本节详细介绍系统关键功能的开发环境与实现逻辑。

\subsubsection{开发与运行环境}

系统基于 Python 生态构建,采用前后端分离模式。硬件环境采用 Nvidia GPU 算力平台;软件选型包括 Python 3.8、PyTorch、PyTorch Geometric(PyG)、Flask、Redis、Celery 等。技术选型见表~\ref{tab:tech_stack}。

\begin{table}[H]
    \centering
    \caption{系统技术选型与开发环境}
    \label{tab:tech_stack}
    \renewcommand{\arraystretch}{1.1}
    \begin{tabular}{l|l}
        \toprule
        \textbf{组件类别} & \textbf{技术选型} \\
        \midrule
        \textbf{开发语言} & Python 3.8 \\
        \textbf{深度学习框架} & PyTorch 1.10, PyTorch Geometric (PyG) \\
        \textbf{通信协议} & RESTful API / gRPC (模型参数传输) \\
        \textbf{后端框架} & Flask (API服务), Celery (异步任务调度) \\
        \textbf{数据存储} & MySQL 8.0 (元数据), Redis 6.0 (热数据) \\
        \textbf{前端框架} & React + TypeScript + ECharts (可视化) \\
        \bottomrule
    \end{tabular}
\end{table}

\subsubsection{联邦协同训练模块实现}

该模块协调分布式客户端与中心服务器进行多轮联邦学习,实现 Broadcast、Local Update、Compress \& Upload、Secure Aggregation 四个阶段的闭环。

\textbf{(1)同步联邦训练流程。} 单轮训练包含:\textbf{模型分发(Broadcast):} 服务端随机选择 $K$ 个客户端,下发全局模型参数 $W_g$ 与全局属性原型 $P_g$。\textbf{本地更新(Local Update):} 客户端基于本地异构子图执行多轮 FedASCL 训练,双视图编码器分别提取结构特征与属性特征,通过最大化互信息优化节点表示。\textbf{压缩与上传(Compress \& Upload):} 训练结束后,客户端经语义感知压缩器处理梯度,仅上传压缩后的梯度更新与本地聚类中心,原始图数据保留在本地。\textbf{安全聚合(Secure Aggregation):} 服务端收集各节点更新,执行加权聚合,更新全局模型版本。

\textbf{(2)仿真引擎实现。} 为验证算法可行性,系统后端实现了高保真仿真引擎。\texttt{DataPartitioner} 模块将 ACM 等数据集切分为多个互不重叠的 Ego-Subgraph,模拟真实环境下的非独立同分布(Non-IID)特性;多线程机制模拟客户端的并发训练与网络延迟。联邦学习仿真引擎架构如图~\ref{fig:simulator_arch} 所示。

\begin{figure}[H]
    \centering
    \includegraphics[width=0.9\textwidth]{images/联邦学习仿真引擎架构图.png}
    \caption{联邦学习仿真引擎架构图}
    \label{fig:simulator_arch}
\end{figure}

\subsubsection{学术资源推荐与冷启动模块实现}

该模块基于训练好的联邦异构图模型为用户提供个性化学术资源推荐,并重点支持冷启动场景。

\textbf{(1)分级推荐策略。} 系统根据用户交互历史判定激活不同推理分支,其逻辑如图~\ref{fig:rec_logic} 所示。对于有丰富交互历史的活跃用户,采用\textbf{结构优先}策略,利用结构视图聚合“用户-论文-会议”等高阶邻居信息,生成用户嵌入并检索 Top-N 相似文献。对于无交互记录的新注册用户,自动激活 FedASCL 的\textbf{属性映射分支},利用用户属性(研究兴趣、机构等)通过映射网络投影到全局原型空间,基于语义相似度检索相关文献,实现零样本冷启动。

\textbf{(2)可解释性增强。} 系统利用动态元路径的注意力权重提供推荐解释。例如,若 \texttt{User-Author-User} 路径权重较高,前端将提示“推荐该学者是因为你们有共同的合作者”。

\begin{figure}[H]
    \centering
    \includegraphics[width=0.9\textwidth]{images/分级推荐策略.jpg}
    \caption{基于属性-结构双通路的冷启动推荐策略流程图}
    \label{fig:rec_logic}
\end{figure}

\subsubsection{通信压缩与隐私模块实现}

\textbf{(1)语义感知压缩器实现。} 系统在客户端上传阶段集成第四章所述的语义感知压缩策略(动态元路径选择器与残差梯度量化),实现逻辑与 4.2--4.3 节一致,在约 20 倍压缩倍数下保持推荐精度并降低通信开销。

\textbf{(2)本地差分隐私机制。} 为增强隐私保护能力,系统对上传的全局原型或梯度施加拉普拉斯噪声(Laplace Mechanism)。通过配置隐私预算 $\varepsilon$ 与敏感度参数,在满足 $(\varepsilon,\delta)$-差分隐私的前提下,防御标签推理与后门攻击,同时控制对模型收敛精度的影响。

% ----------------------------------------------------------------------
\subsection{系统功能测试}
\label{sec:sys_test}

通过黑盒测试验证系统在真实业务场景下的稳定性和可行性。

\subsubsection{主要功能测试}

重点验证联邦控制与推荐服务的逻辑正确性与业务流程闭环。\textbf{联邦控制测试:} 验证全局模型参数的版本迭代(轮次 $t+1$ 与 $t$ 轮参数发生更新)及数据隔离性(服务端无法读取客户端原始图结构)。\textbf{推荐服务测试:} 验证历史交互丰富的老用户请求推荐时,推荐列表与历史偏好领域一致性高;仅有属性信息的冷启动用户请求推荐时,系统基于属性原型返回相关领域文献。测试结果见表~\ref{tab:func_test}。

\begin{table}[H]
    \centering
    \caption{系统主要功能测试用例及结果}
    \label{tab:func_test}
    \renewcommand{\arraystretch}{1.2}
    \resizebox{\linewidth}{!}{
    \begin{tabular}{c|p{6cm}|p{4cm}|c}
        \toprule
        \textbf{测试模块} & \textbf{测试用例描述} & \textbf{预期结果} & \textbf{结论} \\
        \midrule
        \multirow{2}{*}{\textbf{联邦控制}} & 验证全局模型参数的版本迭代 & 轮次 $t+1$ 的参数与 $t$ 轮发生更新 & 通过 \\
        \cline{2-4}
         & 验证数据隔离性 & 服务端无法读取客户端的原始图结构 & 通过 \\
        \midrule
        \multirow{2}{*}{\textbf{推荐服务}} & 历史交互丰富的老用户请求推荐 & 推荐列表与历史偏好领域一致性高 & 通过 \\
        \cline{2-4}
         & 仅有属性信息的冷启动用户请求推荐 & 系统基于属性原型返回相关领域文献 & 通过 \\
        \midrule
        \textbf{系统监控} & 前端实时拉取训练 Loss 与 AUC 指标 & 曲线随训练轮次动态更新且趋势收敛 & 通过 \\
        \bottomrule
    \end{tabular}}
\end{table}

学术资源推荐列表页与用户个人设置页面分别如图~\ref{fig:ui_rec}、图~\ref{fig:ui_settings} 所示,作为推荐服务与个人画像管理功能的展示。

\begin{figure}[H]
    \centering
    \includegraphics[width=0.9\textwidth]{images/论文推荐页面.png}
    \caption{学术资源推荐列表页}
    \label{fig:ui_rec}
\end{figure}

\begin{figure}[H]
    \centering
    \includegraphics[width=0.9\textwidth]{images/个人设置页面.png}
    \caption{用户个人设置页面}
    \label{fig:ui_settings}
\end{figure}

\subsubsection{性能与鲁棒性验证}

\textbf{(1)系统监控看板测试。} 联邦训练状态监控面板实时展示参与训练的客户端数量、当前训练轮次、全局 Loss 与 AUC 等指标随轮次的变化曲线及各客户端在线状态。测试表明,曲线随训练轮次动态更新且趋势收敛,满足监控与调度需求。监控面板如图~\ref{fig:ui_monitor} 所示。

\begin{figure}[H]
    \centering
    \includegraphics[width=0.9\textwidth]{images/中央服务器训练面板.png}
    \caption{联邦训练状态监控面板}
    \label{fig:ui_monitor}
\end{figure}

\textbf{(2)通信开销对比测试。} 在相同数据集与训练轮次下,对比语义感知压缩开启前后端云之间的累计通信流量。结果表明,开启动态元路径选择与残差梯度量化后,通信开销降低约一个数量级(与第四章表~\ref{tab:communication_cost} 一致),验证了压缩模块在系统集成后的可行性。

测试结果表明,系统成功实现了联邦学习的完整闭环,具备模型分发、本地训练、梯度聚合及基于模型的在线推荐能力;在隐私保护与通信效率方面,数据隔离与语义感知压缩均达到设计目标,验证了 FedASCL 框架在工程落地上的可行性。

% ----------------------------------------------------------------------
\subsection{本章小结}
\label{sec:sys_summary}

本章按系统开发的标准流程,阐述了基于 FedASCL 与语义感知压缩策略的联邦推荐系统的设计与实现。首先,从业务与用户两个维度进行了系统需求分析,明确了联邦身份认证、端云协同训练、隐私保障与高效通信等业务需求,以及个人画像管理、训练监控与学术推荐等用户需求。其次,给出了系统总体架构设计,包括自下而上的四层逻辑架构与数据不出域机制。再次,从开发环境、联邦协同训练、学术资源推荐与冷启动、通信压缩与隐私四个模块介绍了关键实现。最后,通过主要功能测试与性能鲁棒性验证,验证了系统在功能完备性、推荐准确性、隐私保护与通信效率方面的表现。该系统的实现验证了从算法到系统的完整链路,说明该方案可在学术推荐场景下实际部署。
